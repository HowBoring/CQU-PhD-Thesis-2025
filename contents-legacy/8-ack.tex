\documentclass[../main.tex]{subfiles}

\begin{document}

\mychapter{致\hskip\ccwd{}\hskip\ccwd{}谢}

% 这里用盲审环境包裹致谢,在开启盲审开关时,环境内部的内容不予渲染。
\begin{secretizeEnv}
	% \vfill
	时光荏苒,岁月如歌。当为这篇博士学位论文画上最后一个句点时,窗外秋雨蒙蒙。回首在重大的十年,却好像有些淡忘。从初入校园的懵懂与憧憬,到探索路上的迷茫与坚持,再到此刻的沉淀与感恩,百感交集。这一程跋涉,山重而水复。绝非我一人所能完成,其间凝聚了太多人的支持与关爱。我常觉得自己是个幸运的人,在每一段时光里,都能遇到良师益友。在此,请允许我致以最诚挚的谢意。

	首先,我要深深感谢我的导师冯亮教授。虽然六年过去,但我仍然清楚记得第一次拜访你时的场景。经年累月,您对学术的热忱与严谨,对学生的关怀与耐心,一直深深感染着我。您是我的导师,也是我尊敬的长辈。我曾和家人说起,如果说我从事科研工作,那么您就是我人生中的最标准的学者榜样。感谢您在我迷茫时给予的指引,在我困顿时提供的支持。您的教诲将伴随我一生。

	感谢 COOL Lab 的各位。回想起与大家共度的时光,只想感叹,有你们真好,让此行一路没有那么孤独。我习惯帮助大家,也受很多人的照顾。因为在 Lab 中,有难得的纯粹,让我舒心而平静。希望大家都能前程似锦,未来可期。

	感谢我的友人。我并非一个善于交际的人,每一位友人,都是弥足珍贵的联系。感谢你们的帮助和陪伴。

	感谢我的家人。父母的爱,如涓涓细流,润物无声。在我求学的这些年里,你们从未给过我任何压力,总是说:“凡事尽力就好,身后还有我们。”这份信任与包容,是我前行路上最坚实的后盾。时常想起你们为我所做的一切,心中满是感激。希望未来能让你们为我骄傲。也接过你们的担子,让我更有动力去承担生活的责任。也感谢芝,在我的夜空中点亮了月光。

	也要感谢那些失败的实验,那些被拒稿的论文,那些在深夜里涌上心头的自我怀疑。是它们让我明白,成长从来不是一条笔直的大道,而是一条蜿蜒曲折的小径。每一次跌倒,都让我离真实的自己更近一步。

	如今,这段旅程即将画上句号。但我知道,这不是结束,而是另一种开始。未来的路还很长,而我已学会,如何在黑暗中为自己点一盏灯。

	窗外的雨还在霏霏而落,而我已准备好,去迎接下一个晴天。


	\vspace*{2em}
	\begin{flushright}
		{\CJKfontspec{STXingkai} \Large 侯博宇} \hspace*{3.5em}
		\\  \hspace*{\fill} \\
		{二〇二五年十月\hspace*{1em}于重庆}
	\end{flushright}
\end{secretizeEnv}

\end{document}