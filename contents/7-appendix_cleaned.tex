\chapter[附\hskip\ccwd{}\hskip\ccwd{}录]{{\heiti\zihao{3}附\hskip\ccwd{}\hskip\ccwd{}录}}

\mysection{作者在攻读博士学位期间的论文目录}
\label{sec:ch6-4-doctoral-publications}

%下面是盲审标记\cs{secretize}的用法,记得去\textsf{main.tex}开启盲审开关看效果:

% \circled{1}已发表论文

% \begin{enumerate}
%     \item \textbf{\secretize{XU X}}, \secretize{LIU K}, DAI P, et al. Joint task offloading and resource optimization in NOMA-based vehicular edge computing: A game-theoretic DRL approach[J]. Journal of Systems Architecture, 2023, 134: 102780. 影响因子: 5.836(2021), 4.497(5年) (中科院SCI 2区,对应本文第三章)
% 	\item \textbf{\secretize{许新操}}, \secretize{刘凯}, 刘春晖, 等. 基于势博弈的车载边缘计算信道分配方法[J]. 电子学报, 2021,49(5): 851-860. (EI 索引,CCF T1类中文高质量科技期刊,对应本文第三章)
% 	\item \textbf{ \secretize{XU X}}, \secretize{LIU K}, XIAO K, et al. Vehicular fog computing enabled real-time collision warning via trajectory calibration[J]. Mobile Networks and Applications, 2020, 25(6): 2482-2494. 影响因子: 3.077(2021), 2.92(5年) (中科院SCI 3区,对应本文第五章)
% \end{enumerate}
{
	\small
	\setlength{\baselineskip}{20pt}
	\begin{enumerate}[label={[\arabic*]}, leftmargin=*]
		% \item \secretize{\textbf{Yin G}}, \secretize{Huang S}, He T, et al. Mirrored EAST: An Efficient Detector for Automatic Vehicle Identification Number Detection in the Wild[J]. IEEE Transactions on Industrial Informatics, 2023. (中科院SCI一区)
		% \item \secretize{\textbf{Yin G}}, Wang Y, Zhang Y, et al. Adversarial Bidirectional Feature Generation for Generalized Zero-Shot Learning Under Unreliable Semantics[C]//Chinese Conference on Pattern Recognition and Computer Vision. Cham: Springer Nature Switzerland, 2022: 639-654.(CCF-C)
		% \item \secretize{\textbf{Yin G}}, Huangfu L, \secretize{Huang S}, et al. Rethinking the Sample Relations for Few-Shot Classification[J]. Image and Vision Computing. (中科院SCI三区,返修中)
    \item Hou B , Feng L , Chen X ,et al.Evolutionary Transfer Neural Architecture Search Across Spaces via Representation Learning[J].IEEE Transactions on Evolutionary Computation, 2025.(中科院SCI一区)
    \item Xiang J , Hou B , Chen X ,et al.House Layout Generation via Diffusion Model with Relative Room Area Ranking[J].2024 International Joint Conference on Neural Networks, 2024.(EI检索)
    \item 
	\end{enumerate}
}

\mysection{作者在攻读硕士学位期间参与的科研项目}
\label{sec:ch6-5-masters-projects}

{
	\small
	\setlength{\baselineskip}{20pt}
	\begin{enumerate}[label={[\arabic*]}, leftmargin=*]
		\item 国家自然科学基金面上项目,少样本学习特征生成与鲁棒性关键技术研究
		      % (No. 62176030)
		\item 重庆市自然科学基金面上项目,文本描述协同的双向生成式少样本学习研究
	\end{enumerate}
}

\newpage

\mysection{学位论文数据集}
\label{sec:ch6-6-dissertation-datasets}

\begin{table}[h]
	% \resizebox{\textwidth}{!}{%
	\begin{tabular}{|cccccccccccc|}
		\hline
		\multicolumn{4}{|c|}{\heiti{关键词}}                                                   & \multicolumn{4}{c|}{\heiti{密级}}          & \multicolumn{4}{c|}{\heiti{中图分类号}}                                                  \\ \hline
		\multicolumn{4}{|c|}{\begin{tabular}{c} 少样本分类; 关系建模; \\ 对比学习; 语义信息表示 \end{tabular}} & \multicolumn{4}{c|}{公开}                  & \multicolumn{4}{c|}{TP}                                                             \\ \hline
		\multicolumn{3}{|c|}{\heiti{学位授予单位名称}}                                              & \multicolumn{3}{c|}{\heiti{学位授予单位代码}}    & \multicolumn{3}{c|}{\heiti{学位类别}}  & \multicolumn{3}{c|}{\heiti{学位级别}}              \\ \hline
		\multicolumn{3}{|c|}{\secretize{重庆大学}}                                              & \multicolumn{3}{c|}{\secretize{10611}}   & \multicolumn{3}{c|}{学术学位}          & \multicolumn{3}{c|}{硕士}                        \\ \hline
		\multicolumn{4}{|c|}{\heiti{论文题名}}                                                  & \multicolumn{4}{c|}{\heiti{并列题名}}        & \multicolumn{4}{c|}{\heiti{论文语种}}                                                   \\ \hline
		\multicolumn{4}{|c|}{\begin{tabular}{c}基于多元关系建模的少样\\本分类算法研究\end{tabular}}           & \multicolumn{4}{c|}{/}                   & \multicolumn{4}{c|}{汉语}                                                             \\ \hline
		\multicolumn{3}{|c|}{\heiti{作者姓名}}                                                  & \multicolumn{3}{c|}{\secretize{尹国伟}}     & \multicolumn{3}{c|}{\heiti{学号}}    & \multicolumn{3}{c|}{\secretize{202124021028t}} \\ \hline
		\multicolumn{6}{|c|}{\heiti{培养单位名称}}                                                & \multicolumn{6}{c|}{\heiti{培养单位代码}}                                                                                            \\ \hline
		\multicolumn{6}{|c|}{\secretize{重庆大学}}                                              & \multicolumn{6}{c|}{\secretize{10611}}                                                                                         \\ \hline
		\multicolumn{3}{|c|}{\heiti{学科专业}}                                                  & \multicolumn{3}{c|}{\heiti{研究方向}}        & \multicolumn{3}{c|}{\heiti{学制}}    & \multicolumn{3}{c|}{\heiti{学位授予年}}             \\ \hline
		\multicolumn{3}{|c|}{软件工程}                                                          & \multicolumn{3}{c|}{计算机视觉}               & \multicolumn{3}{c|}{3年}            & \multicolumn{3}{c|}{\secretize{2024年}}         \\ \hline
		\multicolumn{3}{|c|}{\heiti{论文提交日期}}                                                & \multicolumn{3}{c|}{\secretize{2024年6月}} & \multicolumn{3}{c|}{\heiti{论文总页数}} & \multicolumn{3}{c|}{\pageref{LastPage}}        \\ \hline
		\multicolumn{3}{|c|}{\heiti{导师姓名}}                                                  & \multicolumn{3}{c|}{\secretize{黄晟}}      & \multicolumn{3}{c|}{\heiti{职称}}    & \multicolumn{3}{c|}{教授}                        \\ \hline
		\multicolumn{6}{|c|}{\heiti{答辩委员会主席}}                                               & \multicolumn{6}{c|}{A}                                                                                                         \\ \hline
		\multicolumn{12}{|c|}{\heiti{\begin{tabular}{c} 电子版论文提交格式\\ 文本(\checkmark) 图像() 视频()音频()多媒体()其他()\end{tabular}}}                                                                                                     \\ \hline
	\end{tabular}%
	% }
\end{table}

