\documentclass[../main.tex]{subfiles}

\begin{document}

\mychapter{附\hskip\ccwd{}\hskip\ccwd{}录}

\mysection{作者在攻读博士学位期间的论文目录}



%下面是盲审标记\cs{secretize}的用法,记得去\textsf{main.tex}开启盲审开关看效果:



\begin{secretizeEnv}
	\small\noindent(\textbf{注:}本文作者在成果列表中均以加粗字体标识)
\end{secretizeEnv}

{
\small
\setlength{\baselineskip}{20pt}
\begin{enumerate}[label={[\arabic*]}, left=0pt]
	\secretchoose{
	\item 作者, 导师, CHEN X, et al. Evolutionary Transfer Neural Architecture Search Across Spaces via Representation Learning[J]. IEEE Transactions on Evolutionary Computation, 2025. 影响因子: 12.0 (2024) 14.5 (5年) (中科院SCI一区,对应本文第 \ref{sec:ch4-evolutionary-transfer-nas-heterogeneous-spaces} 章)
	\item 作者, WANG C, CHEN X, et al. Prompt-Distiller: Few-Shot Knowledge Distillation for Prompt-Based Language Learners with Dual Contrastive Learning[C] // IEEE International Conference on Acoustics, Speech and Signal Processing. IEEE, 2023: 1-5. (CCF推荐国际学术会议B类,对应本文第 \ref{sec:ch3-dual-contrastive-distillation-for-few-shot-prompts} 章)
	\item 作者, 导师, WANG Y, et al. KG-MFTO: Knowledge-Guided Multi-Form Optimization for Large Language Model Merging[J]. IEEE Transactions on Pattern Analysis and Machine Intelligence, 2025. 影响因子: 18.6 (2024) 20.4 (5年) (中科院SCI一区,评审中,对应本文第 \ref{sec:ch5-knowledge-guided-multi-form-optimization-for-llm-merging} 章)
	\item 作者, DONG J, 导师, et al. Efficient Two-Stage Evolutionary Search of Convolutional Neural Architectures Based on Cell Independence Analysis[C] // International Conference on Neural Information Processing. Springer International Publishing Cham, 2021: 599-607. (EI检索,CCF推荐国际学术会议C类)
	\item DONG J, 作者, 导师, et al. A Cell-based Fast Memetic Algorithm for Automated Convolutional Neural Architecture Design[J]. IEEE Transactions on Neural Networks and Learning Systems, 2022, 34(11): 9040-9053. 影响因子: 8.9 (2024) 11.1 (5年) (中科院SCI一区,对应本文第 \ref{sec:ch4-evolutionary-transfer-nas-heterogeneous-spaces} 章)
	\item XIANG J, 作者, CHEN X, et al. House Layout Generation via Diffusion Model with Relative Room Area Ranking[C] // International Joint Conference on Neural Networks. IEEE, 2024: 1-8. (CCF推荐国际学术会议C类)
	\item HE L, 作者, DONG J, et al. Two-Stage Neural Architecture Optimization with Separated Training and Search[C] // International Joint Conference on Neural Networks. IEEE, 2023: 1-8. (CCF推荐国际学术会议C类)
	\item WANG S, HUANG Y, 作者. Enhanced Dynamic Vehicle Routing via Knowledge Transfer from Customer Representations[C] // International Conference on Neural Information Processing. Springer Nature Singapore Singapore, 2024: 31-44. (EI检索,CCF推荐国际学术会议C类)
	      }{
	\item \secretize{\textbf{HOU B}}, \secretize{FENG L}, CHEN X, et al. Evolutionary Transfer Neural Architecture Search Across Spaces via Representation Learning[J]. IEEE Transactions on Evolutionary Computation, 2025. 影响因子: 12.0 (2024) 14.5 (5年) (中科院SCI一区,对应本文第 \ref{sec:ch4-evolutionary-transfer-nas-heterogeneous-spaces} 章)
	\item \secretize{\textbf{HOU B}}, WANG C, CHEN X, et al. Prompt-Distiller: Few-Shot Knowledge Distillation for Prompt-Based Language Learners with Dual Contrastive Learning[C] // IEEE International Conference on Acoustics, Speech and Signal Processing. IEEE, 2023: 1-5. (CCF推荐国际学术会议B类,对应本文第 \ref{sec:ch3-dual-contrastive-distillation-for-few-shot-prompts} 章)
	\item \secretize{\textbf{HOU B}}, \secretize{FENG L}, WANG Y, et al. KG-MFTO: Knowledge-Guided Multi-Form Optimization for Large Language Model Merging[J]. IEEE Transactions on Pattern Analysis and Machine Intelligence, 2025. 影响因子: 18.6 (2024) 20.4 (5年) (中科院SCI一区,评审中,对应本文第 \ref{sec:ch5-knowledge-guided-multi-form-optimization-for-llm-merging} 章)
	\item \secretize{\textbf{HOU B}}, DONG J, \secretize{FENG L}, et al. Efficient Two-Stage Evolutionary Search of Convolutional Neural Architectures Based on Cell Independence Analysis[C] // International Conference on Neural Information Processing. Springer International Publishing Cham, 2021: 599-607. (EI检索,CCF推荐国际学术会议C类)
	\item DONG J, \secretize{\textbf{HOU B}}, \secretize{FENG L}, et al. A Cell-based Fast Memetic Algorithm for Automated Convolutional Neural Architecture Design[J]. IEEE Transactions on Neural Networks and Learning Systems, 2022, 34(11): 9040-9053. 影响因子: 8.9 (2024) 11.1 (5年) (中科院SCI一区,对应本文第 \ref{sec:ch4-evolutionary-transfer-nas-heterogeneous-spaces} 章)
	\item XIANG J, \secretize{\textbf{HOU B}}, CHEN X, et al. House Layout Generation via Diffusion Model with Relative Room Area Ranking[C] // International Joint Conference on Neural Networks. IEEE, 2024: 1-8. (CCF推荐国际学术会议C类)
	\item HE L, \secretize{\textbf{HOU B}}, DONG J, et al. Two-Stage Neural Architecture Optimization with Separated Training and Search[C] // International Joint Conference on Neural Networks. IEEE, 2023: 1-8. (CCF推荐国际学术会议C类)
	\item WANG S, HUANG Y, \secretize{\textbf{HOU B}}. Enhanced Dynamic Vehicle Routing via Knowledge Transfer from Customer Representations[C] // International Conference on Neural Information Processing. Springer Nature Singapore Singapore, 2024: 31-44. (EI检索,CCF推荐国际学术会议C类)
	      }
\end{enumerate}
}



\mysection{作者在攻读博士学位期间参与的科研项目}

{
	\small
	\setlength{\baselineskip}{20pt}
	\begin{enumerate}[label={[\arabic*]}, left=0pt]
		\item 科技部. 面向全价值链的一体化协同建模理论、技术与智能算法[R]. 国家重点研发计划课题 (No. 2022YFC3801703), 2022–2025.
		\item 国家自然科学基金委员会. 基于迁移学习的智能多任务高性能优化算法研究[R]. 国家自然科学基金面上项目 (No. 61876025), 2019–2024.
	\end{enumerate}
}

\newpage
\section[\hspace{-2pt}学位论文数据集]{{\heiti\zihao{-3} \hspace{-8pt}学位论文数据集}}

\begin{table}[h]
	% \resizebox{\textwidth}{!}{%
	\begin{tabular}{|cccccccccccc|}
		\hline
		\multicolumn{4}{|c|}{\heiti{关键词}}                                         & \multicolumn{4}{c|}{\heiti{密级}}           & \multicolumn{4}{c|}{\heiti{中图分类号}}                                                \\ \hline
		\multicolumn{4}{|c|}{\begin{tabular}{c} 深度神经网络;高效人工智能;迁移学习 \end{tabular}} & \multicolumn{4}{c|}{公开}                   & \multicolumn{4}{c|}{TP}                                                           \\ \hline
		\multicolumn{3}{|c|}{\heiti{学位授予单位名称}}                                    & \multicolumn{3}{c|}{\heiti{学位授予单位代码}}     & \multicolumn{3}{c|}{\heiti{学位类别}}  & \multicolumn{3}{c|}{\heiti{学位级别}}            \\ \hline
		\multicolumn{3}{|c|}{\secretize{重庆大学}}                                    & \multicolumn{3}{c|}{\secretize{10611}}    & \multicolumn{3}{c|}{学术学位}          & \multicolumn{3}{c|}{博士}                      \\ \hline
		\multicolumn{4}{|c|}{\heiti{论文题名}}                                        & \multicolumn{4}{c|}{\heiti{并列题名}}         & \multicolumn{4}{c|}{\heiti{论文语种}}                                                 \\ \hline
		\multicolumn{4}{|c|}{\begin{tabular}{c}基于知识迁移的深度模型高效构建方法研究\end{tabular}}  & \multicolumn{4}{c|}{/}                    & \multicolumn{4}{c|}{汉语}                                                           \\ \hline
		\multicolumn{3}{|c|}{\heiti{作者姓名}}                                        & \multicolumn{3}{c|}{\secretize{侯博宇}}      & \multicolumn{3}{c|}{\heiti{学号}}    & \multicolumn{3}{c|}{\secretize{20211401002}} \\ \hline
		\multicolumn{6}{|c|}{\heiti{培养单位名称}}                                      & \multicolumn{6}{c|}{\heiti{培养单位代码}}                                                                                           \\ \hline
		\multicolumn{6}{|c|}{\secretize{重庆大学}}                                    & \multicolumn{6}{c|}{\secretize{10611}}                                                                                        \\ \hline
		\multicolumn{3}{|c|}{\heiti{学科专业}}                                        & \multicolumn{3}{c|}{\heiti{研究方向}}         & \multicolumn{3}{c|}{\heiti{学制}}    & \multicolumn{3}{c|}{\heiti{学位授予年}}           \\ \hline
		\multicolumn{3}{|c|}{计算机科学与技术}                                            & \multicolumn{3}{c|}{深度学习}                 & \multicolumn{3}{c|}{4年}            & \multicolumn{3}{c|}{\secretize{2025年}}       \\ \hline
		\multicolumn{3}{|c|}{\heiti{论文提交日期}}                                      & \multicolumn{3}{c|}{\secretize{2025年12月}} & \multicolumn{3}{c|}{\heiti{论文总页数}} & \multicolumn{3}{c|}{\pageref{LastPage}}      \\ \hline
		\multicolumn{3}{|c|}{\heiti{导师姓名}}                                        & \multicolumn{3}{c|}{\secretize{冯亮}}       & \multicolumn{3}{c|}{\heiti{职称}}    & \multicolumn{3}{c|}{教授}                      \\ \hline
		\multicolumn{6}{|c|}{\heiti{答辩委员会主席}}                                     & \multicolumn{6}{c|}{}                                                                                                         \\ \hline
		\multicolumn{12}{|c|}{\heiti{\begin{tabular}{c} 电子版论文提交格式\\ 文本(\checkmark) 图像() 视频()音频()多媒体()其他()\end{tabular}}}                                                                                          \\ \hline
	\end{tabular}%
	% }
\end{table}

\end{document}