\documentclass[../main.tex]{subfiles}
\graphicspath{{../figures/}}

\begin{document}

\mychapter{绪\hskip\ccwd{}\hskip\ccwd{}论}
\label{sec:ch1-preface}

\mysection{研究背景}
\label{sec:ch1-1-research-background-and-significance}

深度学习作为人工智能的核心分支,已成为推动新一轮科技革命与产业变革的关键技术力量\cite{deeplearning_lecun_2015}。在全球范围内,深度学习技术正深刻改变经济社会发展格局。据麦肯锡公司(McKinsey \& Company)发布的评估报告分析,生成式人工智能技术至2030年有望为全球经济贡献约7万亿美元的价值,其中中国将贡献近2万亿美元,占全球总量的三分之一\cite{生成式ai在中国:2万亿美元的经济价值_沈愷_2025}。在产业层面,深度学习技术已广泛渗透至智能制造、医疗诊断、自动驾驶、金融科技等关键领域。以智能制造为例,《中国制造2025》战略规划明确提出“加快工业互联网发展,推动智能装备和产品应用”的目标\cite{国务院关于印发《中国制造2025》的通知_国务院_2015},当前基于深度学习的机器视觉质检系统已在多个行业实现99.8\%以上的准确率,人工成本降低约70\%~\cite{推动人工智能产业迈向更高水平_经济日报_2025}。从国家战略层面来看,人工智能已被确立为国家战略性技术。国务院2017年印发的《新一代人工智能发展规划》首次系统提出"三步走"战略目标\cite{国务院关于印发新一代人工智能发展规划的通知_国务院_2017}:2020年总体技术和应用与世界先进水平同步,2025年部分技术达到世界领先水平,2030年成为世界主要人工智能创新中心。2025年《关于深入实施“人工智能+”行动的意见》进一步细化实施路径,提出构建“技术–产业–应用”协同发展生态,重点突破多模态融合学习算法、神经符号系统等关键技术\cite{国务院关于深入实施“人工智能+”行动的意见_国务院_2025}。人工智能和深度学习技术已成为构建科技强国、推动高质量发展、实现产业优化升级的重要支撑。

深度学习的发展历程以一系列关键的架构革新为标志,这些革新共同推动了人工智能的性能边界。2012年,AlexNet \cite{imagenetclassificationdeep_krizhevsky_2012} 通过深层卷积神经网络(Convolutional Neural Networks,CNN)结构,在 ImageNet 大规模视觉识别挑战赛(ILSVRC)\cite{imagenetlargescale_deng_2009}上取得了突破性成果,验证了深度模型在复杂视觉特征提取上的巨大潜力。随后,2015年提出的残差网络(ResNet)引入残差连接(Residual Connection)机制,有效缓解了深度神经网络训练中的梯度消失问题,使得构建数百乃至数千层的网络成为可能\cite{deepresiduallearning_he_2016}。近年来,以 Transformer \cite{attentionisall_vaswani_2017} 架构为代表的模型,利用自注意力机制(Self-Attention Mechanism)高效捕获长程依赖关系,不仅革新了自然语言处理(Natural Language Processing,NLP)领域,并催生了如 BERT(Bidirectional Encoder Representations from Transformers)~\cite{bertpretraining_devlin_2019} 和 GPT(Generative Pre-trained Transformer)系列 \cite{improvinglanguageunderstanding_radford_2018} 这样的大规模预训练模型。这一演进过程伴随着一个显著特征:模型参数规模的指数级增长。如图~\ref{fig:Ch1-1_model_parameter_evolve} 所示,典型神经网络模型的参数量在短时间内增长了数百万倍。以NLP领域的GPT模型为例,其参数量从GPT-1的1.1亿迅速扩展至GPT-2的15亿、GPT-3的1746亿,直至GPT-4的万亿级别 \cite{gpt4technical_openai_2023}。类似的规模扩张亦体现在计算机视觉领域,例如 Vision Transformer (ViT) \cite{imageisworth_dosovitskiy_2021} 及其后续模型。参数规模的快速膨胀已成为现代深度学习获取高性能的关键途径之一。

\begin{figure}
	\centering
	\includegraphics[width=.8\textwidth]{Ch1-1_model_parameter_evolve.pdf}
	\bicaption[典型人工神经网络模型参数规模的指数级增长趋势]{典型人工神经网络模型参数规模的指数级增长趋势\cite{ourworldindata_ai_parameters,epoch_ai_parameters_2025}。}[Exponential growth of parameters in notable Artificial Neural Networks]{Exponential growth of parameters in notable Artificial Neural Networks\cite{ourworldindata_ai_parameters,epoch_ai_parameters_2025}.}
	\label{fig:Ch1-1_model_parameter_evolve}
\end{figure}

这一轮深度学习的飞速发展,普遍被认为得益于三大关键驱动力的协同推动:大规模数据、模型架构创新,以及计算资源与训练方法的突破。这三者共同构成了深度学习快速发展的核心支柱,被誉为深度学习的“三驾马车”\cite{深度学习研究综述_孙志军_2012}。

首先是数据驱动力的释放。大规模、高质量的标注数据集的出现成为深度学习能够实现飞跃的必要条件。由李飞飞等人建立的 ImageNet \cite{imagenetlargescale_deng_2009}作为最典型的代表,包含超过1400万张高分辨率标注图像,涵盖21841个类别,特别是用于ILSVRC竞赛的ILSVRC2012子集包含1000个类别的128万张训练图像,为深度学习模型提供了丰富知识来源和泛化支撑。这些大规模标注数据集不仅为监督学习提供了坚实的基础,更为迁移学习和预训练范式的发展奠定了基石。除了ImageNet外,MS COCO \cite{microsoftcococommon_lin_2014}、Cityscapes \cite{cityscapesdatasetsemantic_cordts_2016} 等大规模数据集的出现,使得模型能够在更多样化的任务上获得充分的训练信号。近年来,数据集规模进一步扩大,如 LAION-5B \cite{laion5bopen_schuhmann_2022} 包含 50 亿图像-文本对,支持多模态预训练。而 Common Crawl \cite{commoncrawlcorpus_ccf_2025} 则提供了 PB 级别的、累计 3000 亿+ 页面的长期开放网页语料库,为自然语言处理模型提供了海量训练资源。这些数据集的涌现极大地推动了深度学习模型在视觉和语言等领域的性能提升。

其次是架构创新的不断推进。深度神经网络架构的发展经历了从手工设计到自动搜索的转变。在手工设计阶段,研究者通过引入新的结构模块和连接方式来提升模型性能。AlexNet \cite{imagenetclassificationdeep_krizhevsky_2012} 引入的 ReLU 激活函数\cite{deepsparserectifier_glorot_2011}和 Dropout 技术\cite{dropoutsimpleway_srivastava_2014}为后续研究奠定了基础。ResNet \cite{deepresiduallearning_he_2016} 提出的残差连接机制解决了深度网络的优化困难。DenseNet \cite{denselyconnectedconvolutional_huang_2017} 进一步探索了密集连接模式。Transformer \cite{attentionisall_vaswani_2017} 架构的自注意力机制为处理长程依赖关系提供了新的解决方案。近年来,神经网络架构搜索(Neural Architecture Search,NAS)技术\cite{neuralarchitecturesearch_elsken_2019} 的发展使得架构设计过程部分实现了自动化。这些架构创新不仅提升了模型的性能上限,更为后续的模型扩张和功能扩展提供了有力支撑。

第三是计算资源与训练算法的突破。高效训练算法及并行计算硬件的发展,特别是GPU的广泛应用,使得大规模模型参数的高效优化成为可能。GPU的并行计算架构具有数千个计算核心,能够同时处理大量数据,相比CPU而言将模型训练速度提升了数十倍至数百倍。同时,随着分布式训练技术(如数据并行、模型并行、流水线并行)的成熟,研究者们能够利用数百甚至数千个GPU组成的计算集群来训练超大规模模型。此外,Adam优化器\cite{adammethodstochastic_kingma_2015}、批量归一化\cite{batchnormalizationaccelerating_ioffe_2015}、混合精度训练\cite{mixedprecisiontraining_micikevicius_2018}等算法创新也大幅加速了收敛过程,提高了训练的稳定性。

在深度学习性能持续提升的同时,模型规模、架构复杂度与计算量的同步增长导致计算资源与能源消耗呈指数级上升,形成了严峻的可持续发展挑战\cite{混合专家大语言模型的系统与架构优化技术综述_王泽昊_2025}。训练成本的快速攀升是最直观的表现。以GPT-3 \cite{languagemodelsare_brown_2020}为例,其训练过程消耗约1287兆瓦时电力,产生约552吨二氧化碳当量的碳排放,相当于约120辆汽车一年的排放量。GPT-4 \cite{gpt4technical_openai_2023}的训练碳排放估计超过1000吨。据预测,到2030年,人工智能行业的碳排放量可能占全球碳排放的3.5\% \cite{carbonfootprintmachine_patterson_2021,carbonscalingextendingneural_jiang_2025}。这种现象背后反映的是模型性能与构建成本之间的非线性关系。根据 OpenAI 提出的 Scaling Law \cite{scalinglawsneural_kaplan_2020} 研究发现,模型性能的提升遵循幂律关系,即性能增益与参数量、数据量、计算量的增长呈幂次函数关系,而非线性关系\cite{explainingneuralscaling_bahri_2021}。这意味着获得边际性能提升所需的资源投入会快速增加。例如,GPT-3的训练数据包含约3000亿个词元,远超早期模型的数据规模。这种资源需求的超线性增长使得模型规模的无限扩张在经济和环境层面都难以为继。技术普及的门槛提升是另一个重要问题。超大规模模型的训练只有少数拥有充足资金和计算资源的大型科技公司(如OpenAI、Google、Meta等)能够承担。这导致深度学习的前沿研究高度集中,不利于学术界和中小企业的广泛参与。这一趋势与人工智能技术普惠化、民主化的发展目标相悖,也可能加剧技术鸿沟和数字不平等。上述问题的核心可概括为“规模–效率矛盾”:模型性能的提升依赖于规模扩张,而规模扩张带来的资源消耗和环境影响已达到难以持续的程度。这一矛盾正成为制约深度学习技术可持续发展和普惠化应用的核心瓶颈。

造成上述效率困境的根本原因在于,深度模型的构建过程在多个维度上都存在高度的资源依赖,而这些依赖恰与前述三大驱动力相对应。为清晰刻画这一问题,本文引入“深度模型构建”(Deep Model Construction)的概念,将其界定为开发深度学习模型的完整过程,包括从数据准备到模型设计再到参数训练的各个阶段。更进一步地,该概念可以系统地分解为三个核心环节,分别承接了“三大驱动力”所涉及的资源依赖:(1) 数据与知识的获取(对应“数据”驱动力),涵盖数据的采集、标注、清洗与增强等流程;(2) 模型架构的设计(对应“架构”驱动力),涵盖模型结构的设计、自动搜索与实现等内容;(3) 参数的实现与优化(对应“算力”驱动力),涵盖模型参数的学习、优化以及超参数调优等过程。上述三个环节共同构成了深度模型构建的完整工作流。

基于上述分析框架,“规模–效率”矛盾可以形式化地理解为这三个构建环节中效率约束的叠加效应。首先,在数据与知识获取环节,模型构建受到大规模样本获取和人工标注成本高昂的严峻制约。大模型往往需要海量高质量样本来训练,但在许多领域大规模标注数据难以获得,数据稀缺已成为主要的知识瓶颈。其次,在模型架构设计环节,神经网络架构的设计与自动化搜索复杂度极高,已成为结构层面的瓶颈。例如早期的神经架构搜索往往需要数千GPU日的计算开销,令高效探索优化网络结构变得十分困难。再次,在参数训练与优化环节,大规模模型的学习依赖于庞大的计算资源投入,训练一个顶尖模型常需要耗费巨额的算力与时间成本。因此,如何在保证模型性能的同时,全面提升这三个环节的效率,已成为亟需解决的科学问题和研究热点。深度模型的高效构建研究正是要针对上述知识、结构、参数三方面的瓶颈提出解决方案,其重要性随着模型规模的持续膨胀愈发凸显。

为系统性应对上述挑战,学术界正在探索超越“暴力堆叠”的可持续发展路径,其中“知识迁移”(Knowledge Transfer)已成为深度模型高效构建最具潜力的核心指导思想之一。知识迁移旨在通过系统地复用与传递已有的可迁移知识(例如领域数据先验、成熟的网络架构经验或预训练模型的参数),来减少从零开始训练新模型所需的代价。其目标是在模型构建过程中充分利用已有知识,以替代高成本的“从头构建”模式,从而大幅提升构建效率。事实上,深度学习研究范式正逐渐转向“预训练–微调”模式,即先利用海量无标签数据进行模型预训练以获取通用知识表示,再在少量下游任务数据上进行微调快速适配。这一模式显著降低了下游任务的数据需求和计算开销,已成为自然语言处理和计算机视觉等领域的主流实践\cite{bertpretraining_devlin_2019,imageisworth_dosovitskiy_2021,switchtransformersscaling_fedus_2021}。而知识蒸馏作为代表性的知识迁移方法,亦在模型压缩与高效部署中发挥关键作用\cite{distillingknowledgeneural_hinton_2015}。该方法通过将大型“教师模型”的输出分布或中间特征迁移至轻量级“学生模型”,在显著降低模型复杂度的同时保留其泛化能力\cite{patientknowledgedistillation_sun_2019}。此外,神经架构搜索技术的发展也开始借助迁移学习的思想,通过迁移已有任务上验证的优秀架构经验,来加速新任务的架构搜索过程,显著降低搜索成本\cite{neuralarchitecturetransfer_lu_2021,emtnastransferring_liao_2023,evolutionarymultitaskconvolutional_zhou_2024}。这些方法共同体现了知识迁移在提升深度模型构建效率方面的巨大潜力。然而,知识迁移的有效性高度依赖于源任务与目标任务之间的语义相关性与分布一致性。若迁移策略设计不当,可能引发负迁移(negative transfer)现象,即引入的先验知识反而损害目标任务性能 \cite{surveytransferlearning_pan_2010}。此外,不同构建环节对可迁移知识的表征形式、迁移粒度与鲁棒性要求存在显著差异:数据环节关注分布对齐与语义一致性,结构环节强调拓扑可迁移性,参数环节则需解决表示对齐与优化兼容性问题。因此,如何在复杂异构场景下精准识别可迁移知识、设计任务自适应的迁移机制,并建立迁移效果的理论保障,仍是当前研究的关键挑战。

总而言之,知识迁移为缓解规模与效率的矛盾提供了兼具理论价值与实践意义的前沿范式。本文的研究工作即是围绕这一核心思想展开,重点探索如何在模型构建的各个关键环节(知识、结构、参数)中引入知识迁移的理念与方法,以缓解不同维度的效率瓶颈,推动深度学习模型构建朝着更高效可持续的方向发展。

\mysection{国内外研究现状}\label{sec:ch1-2-global-research-status}

基于上述对深度模型构建的界定及其“规模–效率矛盾”的分析,本节将系统回顾国内外的相关研究进展,并遵循“环节解构–瓶颈分析–范式提炼”的逻辑脉络展开论述。
首先,我们将在 \ref{sec:ch1-2-1_key-phase-of-deep-model-construction} 节深入探讨深度模型构建的核心环节,将构建过程解构为数据与知识准备、模型架构设计与实现、参数学习与优化三个核心环节,并阐明贯穿其中的研究路线。
接着,基于此工作流框架,\ref{sec:ch1-2-2_efficiency-construction} 节将承接这三个环节,深度模型构建的效率瓶颈与优化技术逐一分析各自面临的效率挑战,并综述学术界为应对这些瓶颈所采取的关键优化技术。
最后,\ref{sec:ch1-2-3_deep-mode-construction-knowledge-transfer} 节将从前述的各项高效技术中提炼出“知识迁移”这一核心理念,将其作为贯穿并解决各环节效率问题的统一视角,并分析其在高效构建中的关键作用。

\mysubsection{深度模型构建的核心环节}\label{sec:ch1-2-1_key-phase-of-deep-model-construction}

深度模型的构建是一个涉及多阶段、多类型资源的复杂系统工程,围绕数据与知识构造、模型设计原理与方法、以及深度训练范式与算法的相关研究,已共同构成了该概念的核心内涵。现有文献普遍将深度模型的构建过程视为由数据与知识准备、模型架构设计与实现、参数学习与优化共同构成的系统化工作流。图 \ref{fig:three_stage_of_training} 描述了该流程中的主要环节和内容。深度模型构建随着深度学习的发展不断迭代和成熟,已经形成了相对完备的方法体系。值得注意的是,虽然不同任务领域的工作流在实现细节上存在差异,上述“数据 $\rightarrow$ 架构 $\rightarrow$ 参数”的基本组织结构,仍在各领域中具有普遍适用性。本节首先分别回顾这三大环节的主线研究方法。

\begin{figure}
	\centering
	\includegraphics[width=\textwidth]{Ch1-1_three_stage_of_training.pdf}
	\bicaption[深度模型构建的工作流]{深度模型构建的三个主要环节及其内容。}[Three-stage framework for deep model construction]{The three main stages of deep model construction and their respective components.}
	\label{fig:three_stage_of_training}
\end{figure}

数据与知识准备是深度模型构建的起点,决定了模型学习所需的信息基础\cite{imagenetlargescale_deng_2009}。传统深度学习方法高度依赖大规模人工标注数据。此过程通过数据采集、标注、清洗与增强等操作,将原始数据转换为模型适用的输入格式。其中,数据标注是资源消耗最密集、成本最高的环节之一。特别是在医疗影像、自动驾驶等对精度要求严苛的领域,高质量标注的成本常构成项目总预算的显著部分。数据清洗与预处理旨在处理数据质量问题。预处理流水线通常包括格式标准化、归一化、缺失值插补及异常值检测。这些操作对模型最终性能具有显著影响,不当的预处理可能引入偏差、破坏数据分布或丢失关键信息。数据增强则通过对原始数据施加变换(如旋转、裁剪、噪声注入)以扩充训练集,提升模型泛化能力。该方法已被证明是一种有效的隐式正则化手段,但其策略选择与强度调节高度依赖领域知识与实验验证。在知识准备方面,除了数据本身,许多研究还引入外部的先验知识或知识库来辅助模型学习。例如,外部知识图谱、规则库或预训练模型的先验表示常用于弥补有限数据的不足\cite{pretrainedmodels_han_2021}。这一环节的目标是在模型训练前最大程度地整理和提供有用信息,以降低模型学习难度并提升最终性能。

在数据与知识基础之上,架构设计与实现环节决定了模型的结构表达能力、计算特性及其归纳偏置(Inductive Bias)。在深度学习的发展过程中,不同任务催生了诸如卷积神经网络(Convolutional Neural Networks,CNN)\cite{imagenetclassificationdeep_krizhevsky_2012,deepresiduallearning_he_2016,denselyconnectedconvolutional_huang_2017}、循环神经网络(Recurrent Neural Networks,RNN)\cite{learninglongterm_bengio_1994}、基于注意力机制的 Transformer~\cite{attentionisall_vaswani_2017}等典型架构范式。模型架构固有的归纳偏置直接影响模型在特定任务上的学习效率与泛化能力。除宏观范式外,架构内部的信息流拓扑对模型性能亦有决定性影响。例如,He 等人提出的深度残差网络(ResNet)\cite{deepresiduallearning_he_2016},通过引入残差连接(Residual Connections)优化信息流,有效缓解了深度网络中的梯度消失问题,从而实现了信息的更深层次表征。这一思路启发了现代深度学习中模块化设计的原则。现代架构设计普遍遵循模块化原则,通过堆叠卷积层、注意力机制或残差连接等基本单元构建深层网络。这种方式提升了架构的可解释性与可复用性,但也引发了组合爆炸问题,使得手工调优极具挑战,且易导致参数冗余与训练困难。此外,随着硬件的发展,架构设计与硬件的协同优化(Hardware-aware Design)亦是关键。深度可分离卷积\cite{xceptiondeeplearning_chollet_2017,mobilenetsefficientconvolutional_howard_2017}、Flash-Attention~\cite{flashattentionfastmemory_dao_2022}等技术针对硬件特点进行优化,降低了计算与内存开销。为应对上述日益增长的复杂性,或是面向特定场景、特定硬件平台的设计需求,研究者提出了神经架构搜索(Neural Architecture Search,NAS)\cite{neuralarchitecturesearch_zoph_2017}技术,旨在自动化此类设计过程。早期 NAS 依赖强化学习或演化算法等启发式策略,在多种任务上展现了超越人工设计架构的潜力,却也引入了高昂的搜索和评估成本\cite{neuralarchitecturesearch_elsken_2019}。

参数学习与调优环节,旨在将抽象的架构设计转化为具有特定任务能力的可执行模型实例,是深度模型构建的核心步骤。该过程在计算层面涉及两个关键组件:高效的梯度优化器 (Optimizer) 与大规模并行计算硬件。优化器(如带动量的 SGD、RMSProp,及广泛应用的 Adam~\cite{adammethodstochastic_kingma_2015})负责引导模型参数在复杂的损失曲面中高效收敛。并行硬件(以图形处理器 GPU 和张量处理单元 TPU 为代表)则提供了执行大规模矩阵运算所需的算力基础,是实现深度神经网络高效训练的前提。除优化算法与算力外,训练过程的精细化调控对模型最终性能亦有关键影响。为抑制过拟合,Dropout\cite{dropoutsimpleway_srivastava_2014}、权重衰减(L2 Regularization)等正则化技术被广泛用于增强模型的泛化性。同时,精细的训练与验证协议,例如学习率调度 (Learning Rate Scheduling)、早停策略 (Early Stopping) 及交叉验证 (Cross-validation),对于加速收敛和选择最优模型检查点(Checkpoint)至关重要。然而,随着模型架构日趋复杂,参数规模的持续膨胀,使得训练成本成为制约研究与应用深化的严峻挑战。为应对上述挑战,特别是缓解下游任务数据稀缺和高昂训练成本的问题,深度学习的研究范式逐渐转向“预训练-微调”(Pre-training and Fine-tuning)\cite{bertpretraining_devlin_2019}。该范式首先利用海量的无标注或自监督数据对模型进行预训练,使其捕获通用的、可迁移的知识表示(General-purpose Representation)。随后,仅需在特定下游任务的标注数据上进行微调 (Fine-tuning),即可快速适配新任务。这种模式显著降低了对下游任务的数据需求与计算开销,已成为自然语言处理和计算机视觉等领域的主流实践。

总体来看,深度模型构建的方法体系已较为完备,涵盖了从数据到架构再到参数优化的完整流程。然而,在现有文献中,大部分工作以追求性能最优(如提高准确率、降低错误率)为主要目标,对应的研究评测指标多集中于模型的精度和任务效果。而与构建效率相关的维度——例如数据标注成本、训练所需浮点运算次数(FLOPs)、模型参数量、显存和内存占用、训练和推理的延迟、吞吐量、能耗乃至碳排放、总拥有成本(TCO)、以及开发调试周期等——往往被放在次要地位进行讨论或仅在特定环节进行局部优化。而随着模型规模的持续膨胀,这种“性能优先、效率滞后”的现状导致模型构建过程中的资源消耗问题日益突出,越来越多的研究开始转向效率视角,试图在有限资源下提升模型构建的整体效率。这一趋势为后续的深度模型高效构建研究提供了明确的动机与切入点。

\mysubsection{深度模型构建的效率瓶颈及其优化技术}\label{sec:ch1-2-2_efficiency-construction}

针对上述深度模型构建过程中存在的效率挑战,国内外学者开展了多条技术路线的探索,以降低各环节的资源消耗、提高构建性价比。现有研究从数据与知识、模型架构、参数优化三个功能环节入手,提出了多种面向效率提升的方法。本节将分别按这三方面进行综述,并分析各自的效率瓶颈、关键技术,并对现有研究的局限性进行讨论。

\mysubsubsection{数据与知识获取环节的效率化}

面向数据与知识获取环节的效率研究,主要聚焦于在有限标注或计算资源的条件下,最大化数据所蕴含的可学习信息量,并提高模型对外部知识的可迁移利用能力。换言之,此方向的核心问题是数据效率:如何用更少的数据或标注成本达到接近使用大量数据时的模型性能,同时充分利用已有知识来减少重复学习。

在数据效率方面,一系列方法被提出以减少对海量人工标注数据的依赖。少样本学习(Few-Shot Learning)探索模型在只有极少标注样本的情况下泛化到新类别的能力,典型手段包括基于元学习的快速适应和通过度量学习度量新样本与已知类别的相似度\cite{modelagnosticmeta_finn_2017,prototypicalnetworksfew_snell_2017,matchingnetworksone_vinyals_2016}。弱监督学习和半监督学习则利用未标注数据或弱标注(如仅有部分标签、不精确标签),通过生成伪标签、自训练等技术,从未标注数据中挖掘有用信息,从而降低全面标注的成本\cite{pseudolabelsimple_lee_2013}。主动学习进一步提高标注效率:模型主动挑选对自身最有价值的未标注样本请求人工标注,以用尽可能少的标注获得最大的性能提升\cite{surveydeepactive_li_2024}。此外,数据合成方法通过模拟或生成技术合成额外的训练样本\cite{surveydatasynthesis_chang_2024},数据集蒸馏(Dataset Distillation)通过从原始大数据集中“蒸馏”出一个包含少量合成样本的小数据集,使模型在该小数据集上训练即可取得接近用完整数据训练的效果\cite{datadistillationsurvey_sachdeva_2023}。这些方法共同的目标是在数据获取和标注成本一定的情况下,提升模型所能学习到的有效信息量,用有限的数据实现更高的性能。

在模型知识利用方面,知识蒸馏(Knowledge Distillation)为代表的技术通过迁移已有模型或教师模型的知识来提高学习效率\cite{distillingknowledgeneural_hinton_2015,knowledgedistillationsurvey_gou_2021}。经典的教师–学生蒸馏框架中,一个性能较强的教师模型(通常参数规模大、预先充分训练)用于指导一个较小学生模型的训练,学生通过匹配教师的输出分布或特征表示来获得与教师相近的性能\cite{distillingknowledgeneural_hinton_2015,fitnetshintsthin_romero_2015,patientknowledgedistillation_sun_2019}。这一过程中,教师模型所蕴含的知识被高效地传递给学生模型,相当于用教师的经验丰富学生,从而在学生模型参数量更小、训练数据相同甚至更少的情况下,实现性能提升。扩展的多教师蒸馏利用多个教师模型提供多样化的知识来源,自蒸馏则让模型在不借助外部教师的情况下蒸馏自身不同训练阶段或不同子模型的知识\cite{bornagainneural_furlanello_2018,tinybertdistillingbert_jiao_2020,wellreadstudents_turc_2019}。除此之外,表示对齐和特征复用等技术通过在多任务或多模型间共享和复用中间特征表示,达到一份学习成果服务多份任务的效果,从而提高知识使用的整体效率。这类方法的本质是在模型构建过程中重复使用已有的知识成果:要么通过教师模型将知识迁移到新模型,要么在任务间共享表征,避免每次从零开始学习。

然而需要指出的是,上述方法往往附带一定的局限。许多数据高效方法(如少样本、半监督)在实际应用中依赖于数据分布的特定假设或先验,当分布偏离预期或跨领域时性能可能大幅下降。知识蒸馏等手段则高度依赖高质量的教师模型或先验知识的获取,如果教师本身不够强大或不够契合新任务,学生模型的效果提升将十分有限。跨域的鲁棒性也是一大挑战:模型从一种任务迁移到另一种任务时,原有知识是否保持有效难以保证。进一步地,尽管这些方法旨在提高效率,但迁移稳定性和可解释性常常不足——模型为何以及在何种条件下能成功复用知识缺乏清晰理论支撑,使得方法在复杂场景下的可靠性受到质疑。

\mysubsubsection{架构设计环节的效率化}

在深度模型的架构设计与实现方面,提高效率的研究主要关注于有限计算预算下获取高性价比的模型结构及其实现方案。随着模型规模日益增长和应用场景对实时性的要求,如何在保证模型精度的同时尽量降低计算、存储和能耗成本,成为架构研究的重要课题。 一类直接的思路是对已有模型设计进行优化和压缩,以得到更轻量级的高效模型。具体技术包括:模型剪枝通过剪除冗余的网络连接或神经元,仅保留对输出影响较大的部分,从而减少模型参数量和计算量\cite{deepcompressioncompressing_han_2016};模型量化将权重和激活从高精度(如32位浮点)表示转换为低精度(如8位定点),极大降低存储占用和矩阵计算的复杂度,同时借助量化感知训练等策略尽量保持性能不损失\cite{quantizationtrainingneural_jacob_2018};低秩分解和其他结构化分解方法通过将网络中的权重张量分解为低秩近似或特定结构(如用几个小矩阵相乘来近似一个大矩阵),从数学层面减少计算开销\cite{gao2021-jos-compress};设计轻量化算子与瓶颈单元,如MobileNet系列中使用深度可分离卷积\cite{xceptiondeeplearning_chollet_2017,mobilenetsefficientconvolutional_howard_2017},ResNet中引入瓶颈块等\cite{deepresiduallearning_he_2016},以更少的参数实现同等的特征提取能力。这些压缩和优化技术通常在不显著影响模型精度的前提下,将模型尺寸和每次推理所需计算降低一个数量级以上,使模型更易部署在资源受限的设备上或满足实时应用需求。此外,硬件感知的架构设计近年来也受到关注,即在设计模型时将特定硬件(如GPU、TPU、移动端芯片)的计算特性纳入考虑。例如,为张量处理器优化的Transformers架构、适配移动设备的神经网络(MobileNet、EfficientNet等),都体现了根据硬件长处定制架构以最大化效率的思想。

另一类重要方向是自动化的模型架构搜索,期望通过算法来发现兼顾性能和效率的架构。NAS 技术利用强化学习、演化算法或梯度优化,在给定的算力预算内搜索出最优的网络结构\cite{neuralarchitecturesearch_elsken_2019}。然而需要注意的是,尽管 NAS 等方法提供了自动化的手段,但自动化并不等于高效:NAS 需要在搜索过程中对搜索得到候选架构进行持续的评估,本身计算开销巨大,在搜索过程中耗费的算力甚至远超训练单个模型所需。例如 NASNet~\cite{learningtransferablearchitectures_zoph_2018} 和 AmoebaNet~\cite{regularizedevolutionimage_real_2019} 作为早期的尝试,分别耗费了 2000 和 3150 GPU 天的计算资源。为了应对高昂的搜索和评估成本,Bender 等人提出的一站式搜索(One-Shot NAS)\cite{understandingsimplifyingone_bender_2018} 和 Pham 等人提出的 ENAS~\cite{efficientneuralarchitecture_pham_2018},通过权重共享手段,试图在单次训练过程中评估大量候选架构,从而降低搜索的成本。Liu 等人提出的 DARTS~\cite{dartsdifferentiablearchitecture_liu_2019} 则在权重共享的基础上,进一步将架构搜索转化为了连续可微的优化问题,通过梯度优化方法实现了高效的搜索。此外,代理模型或性能预测器被用于加速架构评估,即通过训练一个预测模型来迅速估计候选架构的准确率或其他指标,以减少对每个架构进行完整训练的开销\cite{neuralpredictorneural_wen_2020,endendperformance_sun_2023,generalpurposetransferable_han_2023,renasrelativisticevaluation_xu_2021}。还有工作探索跨任务的可迁移架构设计,即利用在一类任务上搜索得到的优秀架构来指导新任务的模型设计,或者构建元架构先验以缩小新的搜索空间范围\cite{neuralarchitecturetransfer_lu_2021,archgraphacyclic_huang_2022,emtnastransferring_liao_2023,evolutionarymultitaskconvolutional_zhou_2024}。这些方法都旨在缓解人工设计架构的低效,使模型设计过程更自动、更高效。最近的研究开始关注可迁移的 NAS 方法,试图通过在多个任务或数据集上联合搜索,获得更通用的高效架构,从而减少每次新任务都需重新搜索的计算负担。然而,尽管上述方法在一定程度上提升了架构设计的效率,但仍存在显著局限。首先,许多自动化架构搜索方法的计算开销依然很高,尤其是在大规模数据集和复杂任务上,搜索过程中的资源消耗可能远超手工设计的成本。其次,自动化方法往往依赖于预定义的搜索空间,这限制了其探索能力和创新性,可能错过一些非传统但高效的架构设计。此外,不同任务和硬件环境往往对应不同的最佳架构(即存在搜索空间的异质性),这使得一个任务上获得的架构或经验难以直接复用到另一个任务,限制了自动架构设计的普适性。这些问题表明,在架构效率化研究中,如何控制搜索成本、提高跨场景的复用性和结果可靠性,仍然需要进一步的探索。

\mysubsubsection{参数实现与优化环节的效率化}

在参数实现环节,研究重点转向如何降低训练成本并提高参数复用的性价比。这一环节涵盖了模型参数从初始训练、后续微调到跨任务复用的整个生命周期。随着预训练模型规模越来越大、训练代价高昂,以及下游任务层出不穷,需要频繁微调模型以适应新任务,参数效率化的方法应运而生。

一个主要的研究领域是预训练–微调范式与参数高效微调(Parameter-Efficient Fine-Tuning,PEFT)。针对大模型在下游任务上的适配问题,传统做法是对全部参数进行微调,但这在存储和计算上成本巨大。于是,各种参数高效微调方法被提出,在保持预训练模型大部分参数不变的情况下,仅调整很小一部分参数来完成新任务,显著降低了微调成本\cite{parameterefficientfine_han_2024}。其中典型的策略包括:在模型的部分层插入Adapter模块\cite{adaptersunifiedlibrary_poth_2023},小规模参数的瓶颈层通过学习调整特定特征,从而免去修改原模型的大部分权重;通过Prompt Tuning\cite{powerscaleparameter_lester_2021} 向模型输入添加可学习的提示向量,从而影响模型对下游任务的表征,无需修改原有权重;以及LoRA(Low-Rank Adaptation)\cite{loralowrank_hu_2022} 方法对预训练权重施加低秩增量的更新。这些PEFT技术的共同特点是在保证预训练模型基本能力的前提下,以极小的训练参数修改实现对新任务的定制。其优点显而易见:需要更新的参数规模通常不到原模型的1\%甚至更低,微调所需的计算FLOPs和显存开销也大幅减少,使得在普通硬件甚至移动设备上对超大模型进行任务适配成为可能。同时,由于保留了原模型的大部分参数不变,这类方法在工程上易于“落地”,能够快速部署多个任务的定制模型而不必为每个任务维护完整模型副本。然而,相应的缺点也需要权衡:一些方法(如Adapter、Prompt)的引入在推理阶段带来了额外的运算开销或模型复杂度,例如需要额外的前向计算模块或更长的输入,使推理速度有所下降;另外,多任务多Adapter的管理会产生碎片化管理的问题——当一个预训练模型衍生出众多下游任务适配版本时,如何有效管理这些额外参数、以及在推理服务中按需加载,成为新的挑战。

此外,研究者开始探索完全免训练的模型能力集成范式,即模型融合\cite{modelmergingllms_yang_2024}。模型融合技术通过直接操作和组合已训练的参数来应对新任务或新需求。这类方法试图进一步降低计算开销,理想情况下无需额外训练即可得到可用模型。例如,参数平均的方法简单地对同构模型的对应权重取平均,以融合多个模型的知识(典型如在联邦学习中聚合多客户端模型,或在模型集成中平均多个epoch的权重得到更稳定的模型);模型汤(Model Soup)是最近提出的一种模型融合策略,研究者发现对同一模型训练过程中的多个训练检查点进行加权平均,有时能得到比单一检查点性能更好的模型\cite{modelsoupsaveraging_wortsman_2022};此外还有任务算术融合(Task Arithmetic),将模型在下游任务上微调训练产生的参数偏移作为代表下游任务知识的向量,直接对多个任务向量进行线性组合\cite{editingmodelstask_ilharco_2023}。这些融合方式本质上都是利用已有模型的参数作为素材,通过某种组合算则产生新的模型。为了应对不同来源模型参数之间可能存在的冲突,研究者也提出了冲突修正策略,例如 Yadev 等人提出的 TIES-Merging 对冲突的权重采用掩码筛除\cite{tiesmergingresolving_yadav_2023} ,而 Abika 等人则进一步通过演化算法对来自不同模型的权重进行加权调整\cite{evolutionaryoptimizationmodel_akiba_2025},确保融合后的模型在各目标任务上性能不至于显著下降。免训练复用和融合的最大吸引力在于快速和低成本,由于不进行或只进行极少的训练,这种方式能够在几乎不增加额外算力开销的情况下,生成适应新任务或综合多任务能力的模型。在实际应用中,这意味着可以更灵活地组合模型能力,而不必从头训练。然而,这类方法目前也存在明显的局限。首先,简单的参数融合容易引发参数冲突与性能退化:不同模型参数在同一位置可能含义迥异,直接平均或拼接可能破坏原有训练过的特定结构,使性能不可预测地下降。虽然有掩码等修正手段,但选择哪些参数冲突、如何修正仍缺乏通用准则,需要依赖经验调节。最后,在面对多任务或跨领域的场景下,融合所得模型的一致性和稳定性难以保证,跨任务的一致性不足表现为可能在一种任务上性能提升,却在另一种任务上性能剧烈下降,或者性能随着融合的模型增加而难以调优。因而,在复杂现实需求下,如何确保这种免训练融合既保持各部分能力又不互相干扰,是尚未解决的问题。

总体而言,参数高效微调和模型融合各有侧重:前者偏向保障单任务适配时的效率,后者追求跨模型或跨任务的整合。但三类路径(数据与知识、架构、参数)尽管分别缓解了不同环节的效率压力,整体上仍呈现出局部化、经验驱动、度量不统一的特征。换句话说,目前每类方法多是各自为战,在特定子问题上取得了进展,但缺乏从深度模型构建全流程出发的系统整合和统一评价。各方法之间的关系、有无可能形成协同效应、其贡献在整个构建流程中的占比等问题,仍缺少深入研究。

\mysubsection{基于知识迁移的深度模型高效构建}\label{sec:ch1-2-3_deep-mode-construction-knowledge-transfer}

综观上述各环节的效率化研究,可以发现“知识迁移”思想贯穿于数据、架构、参数三个构建环节,并形成了若干具有代表性的研究方向。迁移学习原本关注的是模型在不同任务或领域间性能的迁移,而在提高构建效率的背景下,迁移的意义在于将已有的表征、经验、能力,抽象为可复用的知识,在新任务中加以复用。基于知识迁移理念的效率研究在各环节呈现以下特点:

在数据与知识视角,迁移思想体现为知识在不同任务间的迁移与复用。无论是知识蒸馏中教师模型对学生模型的“指导”,还是多任务学习中相关任务间的共享表示,又或是通过预训练模型将通用知识应用于下游任务,这些做法都以不同形式实现了领域知识的迁移复用。上述方法的共同优势在于显著提升了数据利用效率(Data Utilization Efficiency)并降低了对显式标注数据(Explicitly Labeled Data)的依赖。例如,在知识蒸馏中,学生模型得以继承教师模型从海量数据中习得的隐式知识(Implicit Knowledge)\cite{distillingknowledgeneural_hinton_2015};在预训练范式中,下游任务模型可在预先训练获取的通用表示基础上进行微调(Fine-tuning),而非从零开始训练(Train from scratch)\cite{bertpretraining_devlin_2019}。这些机制均有效减少了模型对大规模、任务特定型(Task-specific)新数据的依赖,从而在整体上提升了模型的构建与迭代效率。然而,迁移方法的有效性往往依赖于源任务与目标任务之间的相关性(Task Relatedness)。当两者差异较大时,迁移效果可能不佳,甚至出现负迁移(Negative Transfer),即迁移反而降低了目标任务的性能\cite{knowledgedistillationsurvey_gou_2021}。此外,虽然知识迁移显著降低了对标注数据的需求,但仍需一定量的目标任务数据进行微调或适应,以确保迁移知识的有效应用。这也限制了如知识蒸馏等方法在极端数据稀缺场景下的适用性。

在架构设计方面,迁移的思路则体现为设计经验和评价能力在不同任务和架构空间的迁移。通过利用在一个领域中发现的架构模式并将其应用于相关任务,可迁移神经架构搜索(Transferable Neural Architecture Search, TNAS)能够简化神经架构的优化过程\cite{neuralarchitecturetransfer_lu_2021,archgraphacyclic_huang_2022}。这种效率的提升是通过在搜索过程中引入偏置实现的,而该偏置则源于从先前已解决的神经架构搜索 NAS 任务中获得的经验。Liao 等人提出的 EMT-NAS~\cite{emtnastransferring_liao_2023} 进一步强化了这一理念,它采用演化算法来增强灵活性并缓解负迁移问题,从而取得了显著进展。但是 TNAS 的研究仍然受限于统一的架构搜索空间,这限制了其在更广泛任务和架构类型上的适用性\cite{emtnastransferring_liao_2023,evolutionarymultitaskconvolutional_zhou_2024}。近期一些研究致力于开发可迁移的性能预测器,例如 Liu 等人则受域自适应方法启发提出了跨域预测器(Cross-Domain Predictor,CDP),将在小规模搜索空间训练得到的性能预测器应用到更大空间,以预测新架构的性能\cite{bridgegaparchitecture_liu_2022}。然而,对于直接跨越不同架构搜索空间的迁移,目前仍缺乏有效的方法和理论支持。这表明在架构设计环节,尽管迁移思想已被证明能提升设计效率,但其应用范围和稳定性仍有待进一步拓展和验证。

在参数实现环节,迁移主要体现在参数的选择性继承与融合上。预训练–微调本身就是迁移学习的典型应用:将源任务学得的参数作为初始化迁移到目标任务,再通过少量训练适应新任务\cite{bertpretraining_devlin_2019,gpt4technical_openai_2023}。这相对于随机初始化大大减少了训练时间。此外,前述免训练模型融合方法把多个任务的参数直接合并,代表了对于参数的跨模型迁移。它不通过训练迭代,而直接整合已有模型的参数,使得多份知识在一个模型中共存。然而,在多任务融合时,参数冲突问题尤为突出,不同任务的参数可能在同一位置存在矛盾,直接融合可能导致性能下降\cite{tiesmergingresolving_yadav_2023}。尽管已有一些冲突修正策略被提出\cite{tiesmergingresolving_yadav_2023,languagemodelsare_yu_2024},但如何系统地解决跨任务参数冲突仍是一个开放问题。一些工作如 EOMMR~\cite{evolutionaryoptimizationmodel_akiba_2025} 试图通过演化算法对模型参数进行细致的缩放调整,以进一步缓解这个问题,然而随着融合模型数量增加,对应于不同任务的参数之间的冲突复杂性也几何上升,复杂的模型融合问题依然难以解决。

尽管以迁移为核心思想的方法在提升构建效率方面展现了潜力,但目前来看仍然存在一些共性不足。例如在跨任务、跨领域甚至多模型的复杂场景下,迁移策略的稳定性、可解释性与可复现性仍显不足。很多迁移方法在单一源和目标任务对上有效,但当面对多个来源或需要迁移到性质差异很大的任务时,效果可能不稳定甚至失败。综上,上述观察表明迁移思想在提升深度模型构建效率方面确实蕴含潜力,但是在仍然存在亟待解决的问题。

\mysection{研究内容与主要贡献}\label{sec:ch1-4-core-questions-and-contributions}

\mysubsection{研究目标与思路}

本文的核心研究目标是:以知识迁移作为核心指导思想,系统性地探索与实践在深度模型构建中,数据和知识获取、结构设计与参数实现与优化三个关键环节中提升构建效率的方法论。承接前文节的背景分析,当前深度学习领域正面临严峻的“规模–效率”矛盾,这一瓶颈具体表现为在上述三个构建环节中叠加的效率约束。同时,对国内外研究现状的回顾表明,尽管以知识迁移为导向的各类高效构建方法已展现出巨大潜力,但它们在具体实践中仍面临诸多亟待解决的局限性。具体而言:

在数据与知识获取环节,虽然知识蒸馏等迁移技术为降低数据依赖提供了有效途径,但其在少样本场景下的稳定性问题依然突出。有限的样本无法覆盖输入空间,导致学生模型易陷入过拟合和偏差。如何设计鲁棒的迁移机制,在标注数据极度稀缺时,依然能够高效、可靠地传递教师模型的知识,特别是随着大规模预训练模型的兴起,对于提示学习这类新兴范式,仍是一个重要的研究机遇。

在架构设计环节,NAS 提供了自动化设计的能力,但其高昂的计算成本和架构经验难以跨任务复用的问题,尤其是在面对异构搜索空间时的迁移失效,严重制约了其广泛应用。开发能够打破搜索空间壁垒、实现通用架构知识迁移的新型 TNAS 方法,是提升自动化模型设计效率的关键所在。

在参数实现与优化环节,免训练的模型融合范式展现出极致的效率潜力,整合已有模型的能力。然而在融合多个、特别是面向不同任务模型时所面临的参数冲突、性能退化等稳定性挑战,使得这一极具吸引力的技术路径在实践中受到诸多限制。研究稳定、高效且可扩展的多模型参数融合机制,对于实现模型能力的即时按需组合具有重要价值。

基于上述挑战,本文的总体研究思路是采取分而治之的策略。本文将“基于知识迁移的深度模型高效构建”这一宏观目标,分解为三个分别聚焦于数据与知识获取环节、架构设计环节以及参数实现与优化环节的关键科学问题。并针对各个环节的核心瓶颈,探索以知识迁移为指导的创新解决方案,最终旨在为深度学习的高效、可持续发展路径贡献有益的探索。下一节将提出本文在各个层面上所聚焦的具体关键科学问题,以及对应的研究内容。

\mysubsection{关键科学问题}
\label{sec:ch1-4-1-core-scientific-question}

本节进一步将基于知识迁移的深度模型高效构建的研究目标凝练为以下三个分别对应知识、结构与参数层面的关键科学问题。这些问题不仅直接回应了当前深度模型构建面临的效率瓶颈,也代表了其中各个环节在各自方向上亟待突破的前沿挑战。同时,本节将提出针对每个科学问题的具体研究内容,明确本文后续章节的研究重点与创新方向。

\textbf{(Q1)}第一个关键科学问题关注数据与知识获取环节的迁移效率与稳定性,特别是在数据稀缺条件下的挑战。正如 \ref{sec:ch1-2-3_deep-mode-construction-knowledge-transfer} 节所述,尽管知识蒸馏等技术为利用先验知识提供了途径,但在少样本场景下,其效果往往因教师偏差、学生过拟合以及隐性知识难以有效传递等因素而大打折扣。该问题影响了模型的构建效率,更触及了知识表示与迁移的根本机制:当监督信号极其微弱时,模型如何区分并吸收真正有价值的知识,而非噪声或偏差?因此,本研究提出:在缺乏大量标注数据的条件下,如何设计高效且稳定的知识迁移机制,特别是探索新型知识蒸馏策略(例如结合对比学习或利用无标签数据),以从根本上克服少样本场景下的性能退化问题,并确保知识传递的鲁棒性?回答这一问题,不仅能显著降低模型构建对昂贵标注数据的依赖,也将深化对低资源学习环境下知识表示与泛化机理的理解。

\textbf{(Q2)}第二个关键科学问题聚焦于架构设计环节的知识复用效率,特别是在架构范式存在差异时的障碍。\ref{sec:ch1-2-3_deep-mode-construction-knowledge-transfer} 节的分析表明,NAS虽能自动化设计,但其高昂成本和一次性搜索模式限制了效率,而现有的可迁移 NAS 又大多局限于同构搜索空间,难以应对跨越不同架构范式(例如采用不同算子、或具有不同架构约束)的场景。这引出了一个核心的科学挑战:是否存在一种通用的架构知识表示方法,能够超越具体的算子和拓扑细节,捕捉架构设计的本质规律?如果存在,如何基于这种表示实现跨越异构搜索空间的有效知识迁移?因此,本研究提出:在计算资源有限的情况下,如何建立统一的神经架构表示学习框架,并设计相应的跨域迁移策略,以实现跨越不同架构范式(即异构搜索空间)的结构知识有效迁移,从而显著提升NAS的效率与通用性?解决这一问题,不仅能大幅降低自动化模型设计的成本,更有望揭示不同架构家族之间潜在的设计共性与演化联系。

\textbf{(Q3)}最后一个关键科学问题则着眼于参数实现与优化环节的免训练集成,旨在解决多模型能力组合的稳定性难题。\ref{sec:ch1-2-3_deep-mode-construction-knowledge-transfer} 节指出,免训练的模型融合作为一种极具吸引力的能力集成方式,因参数语义错位、非线性干涉等固有难题,在融合多个、特别是面向不同下游任务的模型时,其稳定性和性能保持面临严峻挑战。这触及了深度模型参数空间几何特性以及多模型知识兼容性的基础问题:不同模型学习到的知识如何在参数层面进行有效的、非破坏性的叠加?是否存在一种无需梯度优化的机制,能够智能地识别并缓解参数冲突,保留各模型的核心能力?因此,本研究提出:在免训练条件下,如何设计系统化的参数融合框架,特别是引入显式的知识建模与优化策略,以实现多个下游专长模型间参数的高效、稳定融合,确保在有效集成各模型能力的同时,最大限度地避免性能损失?攻克这一难题,不仅能为即时、低成本地构建具备复合能力的强大模型开辟新途径,也将促进对大型模型参数空间结构及其组合规律的认识。

这三个核心科学问题共同构成了本文研究的出发点和落脚点。他们分别从数据知识、模型架构和模型参数三个维度切入深度模型高效构建的核心挑战,不仅具有重要的实践价值,也蕴含着深入探索深度学习内在机理的理论意义。本文后续章节提出的创新性解决方案,正是围绕这三个核心问题展开的直接回应。

\begin{figure}
	\centering
	\includegraphics[width=\linewidth]{research_question2.pdf}
	\bicaption[核心科学问题与研究目标]{本文围绕深度模型构建的知识、结构与参数三个层面,提出的核心科学问题及其相应的研究内容}[Core questions and contents]{Core scientific questions and research contents across the knowledge, structure, and parameter dimensions of deep model construction}\label{fig:research_questions}
\end{figure}

\mysubsection{本文的贡献}

围绕上述三个核心科学问题,本文分别在深度模型构建的数据与知识、结构设计与参数实现三个层面展开了深入研究,并提出了一系列旨在提升构建效率的创新性方法。这些工作共同构成了本文对“基于知识迁移的深度模型高效构建”这一核心命题的具体探索与实践。本文的主要研究内容和贡献可概括如下:

\textbf{(1)基于双教师对比学习的少样本知识蒸馏}\;针对知识层面在少样本条件下知识迁移效率与稳定性的核心问题(Q1),本文提出了一种双教师对比学习蒸馏框架 (Prompt-Distiller)。基于对少样本场景特点的分析,结合提示学习技术,本文提出同时融合来自提示微调教师的任务特定知识与来自原始预训练教师的通用语言知识,通过多源的知识迁移最大化学生模型的知识输入。为了进一步提升所迁移知识的密度,本文提出了一种基于探针的对比学习策略,以增强对教师模型中间层隐性知识的迁移。此项工作(详见第 3 章)旨在为低资源场景下的模型高效构建提供一种鲁棒的知识迁移解决方案。

\textbf{(2)基于表征学习的跨空间可迁移神经架构搜索}\;为应对结构层面跨异构搜索空间进行架构知识迁移的挑战(Q2),本文提出了一个跨空间的神经架构表示学习与迁移框架(\textsc{Bridge} )。该框架的核心在于突破了现有可迁移 NAS 局限于同构搜索空间的限制。通过设计定制化的神经架构分词器与基于 Transformer 的变分自编码器,能够将来自不同搜索空间的架构映射到统一的、结构感知的潜在表示空间。在此基础上,本文进一步设计了跨域表示映射学习策略与演化序贯迁移优化 算法,能够将源任务中发现的优秀架构经验显式地迁移并适应到目标任务的异构搜索空间中。实验结果表明(详见第三章),\textsc{Bridge} 框架能够在显著降低 NAS 搜索成本的同时,发现与从零搜索相当甚至更优的高性能架构,为实现跨任务、跨架构范式的自动化模型设计经验复用提供了可行路径。

\textbf{(3)知识图谱引导的多形式优化模型融合}\;针对参数层面免训练模型融合的稳定性难题(Q3),本文提出了一种利用模型关系知识图谱指导的多形式优化范式(KG-MFTO)。该方法将复杂的多模型融合问题分解为一系列更易于求解的子问题(形式),并创新性地引入动态演化的知识图谱来显式建模和记录模型间的协同与冲突关系,并通过图注意力网络(Graph Attention Networks,GAT)嵌入为连续表征。为了高效地对知识图谱进行填充,本文通过课程规划器自适应地选择优化形式序列,并利用知识图谱引导的基于 CMA-ES 的演化求解器高效搜索融合参数。将求解得到的形式解将更新至知识图谱,并对图嵌入学习器 GAT 进行监督训练。通过上述“感知-计划-执行-学习”的闭环,从而在无需任何额外训练的条件下,稳定、高效地整合多个下游专长模型的能力。在融合多个 LLM 专长模型时,能够有效缓解参数冲突,显著提升融合后模型在多任务上的综合性能与稳定性,优于现有的静态融合或无指导搜索方法,为实现模型能力的即时、低成本集成提供了一种全新的、基于知识迁移优化的解决方案。

综上所述,本文通过在知识、结构、参数三个层面分别提出的 Prompt-Distiller、\textsc{Bridge} 和 KG-MFTO 方法,为深度模型的高效构建提供了具体的技术支撑。图~\ref{fig:research_questions} 直观地展示了本文围绕深度模型构建的知识、结构与参数三个层面,提出的关键科学问题及其相应的研究内容。这些贡献不仅在各自领域内拓展了知识迁移的应用边界,也共同印证了以知识迁移思想应对模型构建效率瓶颈的有效性与潜力。

\mysection{论文结构安排}\label{sec:ch1-5-thesis-structure}

\begin{figure}[t]
	\centering
	\includegraphics[width=\linewidth]{overall_struction-crop.pdf}
	\bicaption[论文结构示意图]{本文整体结构安排示意图}[Thesis structure]{Overall structure of this thesis}\label{fig:overall_structure}
\end{figure}

本文围绕“基于知识迁移的深度模型高效构建”这一核心命题,分别在知识、结构与参数三个维度展开研究 。全文共分为六个主要章节,如图 \ref{fig:overall_structure} 所示,其结构安排如下:

第1章\; 绪论。 本章首先阐述了研究背景,指出了深度学习发展中日益严峻的规模-效率矛盾,并将其解构为知识获取、结构承载与参数实现三个关键环节的效率瓶颈。随后,提出了本文以知识迁移思想应对这些瓶颈的核心研究命题与三个具体研究目标 。接着,通过回顾国内外相关研究现状,进一步明确了本文研究的切入点与创新性。最后,凝练了核心科学问题、概述了主要贡献,为全文奠定了研究基础和论述框架。

第2章\; 理论基础与预备知识。本章旨在为后续章节深入探讨深度模型高效构建的具体方法奠定必要的理论基础。本章将界定贯穿全文的核心理念——知识迁移,并分别对后续章节重点关注的三个关键问题领域,即少样本学习与知识蒸馏、神经架构搜索、以及模型参数融合,给出形式化的问题定义。

第3章\; 高效数据知识获取:少样本知识蒸馏。本章聚焦于知识层迁移研究,详细阐述了面向少样本提示学习模型的双重对比知识蒸馏方法(Prompt-Distiller)。该章节将深入探讨其理论框架、多源教师蒸馏和对比探针的设计,并通过大量实验验证其在低资源条件下提升知识迁移效率与稳定性的有效性。

第4章\; 高效架构设计:跨空间可迁移神经架构搜索。本章转向结构层迁移研究,提出了旨在解决跨异构搜索空间架构知识迁移难题的 \textsc{Bridge} 框架。 本章将重点介绍其统一神经架构表示学习机制、跨域表示映射方法以及演化序贯迁移优化策略,并通过理论分析与实验评估,展示 \textsc{Bridge} 在实现高效、通用 NAS 方面的能力。

第5章\; 高效参数实现:知识图谱引导的多形式优化模型融合。本章深入探讨参数层迁移研究,提出了面向大语言模型参数融合的知识引导多形式优化方法(KG-MFTO)。该章节将详细介绍如何利用知识图谱、课程规划与知识引导的演化求解器,在零训练条件下实现多个 LLM 参数的高效、稳定融合,并验证其性能与效率优势。

第6章 总结与展望。 本章将对全文的研究工作进行系统性总结,再次凝练本文在数据与知识获取、结构设计与参数实现三个层面所提出的主要方法、核心贡献及其意义。同时,本章也将客观分析当前研究存在的局限性,并基于此对未来可能的研究方向进行展望。

\end{document}
