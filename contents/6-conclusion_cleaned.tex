\documentclass[../main.tex]{subfiles}

\begin{document}

\mychapter{总结与展望}\label{sec:ch6-summary-and-outlook}

\mysection{全文总结}
\label{sec:ch6-1-overall-summary}

本文的研究聚焦于当前深度学习领域面临的核心挑战之一:模型性能的持续提升与日益增长的构建成本(包括数据、算力与人力投入)之间的规模-效率矛盾 。为应对这一挑战,本文提出并系统探索了以知识迁移作为核心指导思想来实现深度模型高效构建的方法论。研究认为,通过有效复用和传递蕴含在已有数据、架构设计经验及已训练模型参数中的先验知识,有望显著降低模型构建在各个环节的资源消耗,推动深度学习向更高效、更可持续的方向发展 。

基于对深度模型构建过程的分析,本文将研究视角划分为知识获取、结构承载与参数实现三个关键维度 。不同于构建统一框架的思路,本文采取了在每个维度上分别独立地应用知识迁移思想进行深入探索的研究路径 。具体而言,本文围绕这三个维度展开了以下主要研究工作:

在知识层面,针对低资源(少样本)场景下模型构建对标注数据的高度依赖问题,本文研究了高效稳定的知识蒸馏机制。第三章提出了 Prompt-Distiller 方法,通过融合双教师知识与对比学习策略,实现了在少样本提示学习设定下的鲁棒知识迁移,有效提升了数据匮乏环境下的模型构建效率 。

在结构层面,为解决神经架构搜索的高昂计算成本与架构经验难以跨越异构搜索空间进行复用的瓶颈,本文探索了跨域架构知识迁移方法。第四章提出了 \textsc{Bridge} 框架,通过构建统一的架构表示学习机制与进化迁移优化策略,实现了异构搜索空间之间的架构设计知识的高效传递,显著降低了自动化模型设计的开销 。

在参数层面,聚焦于免训练模型融合中存在的稳定性与性能保持难题,本文研究了知识引导的参数集成新范式。第五章提出了 KG-MFTO 框架,利用知识图谱与多形式优化策略,在零训练条件下实现了多个大型语言模型参数的高效、稳定融合,为模型能力的即时、低成本集成提供了新的解决方案 。

综上所述,本文通过在知识、结构、参数三个层面的独立研究,分别提出了针对性的、基于知识迁移思想的高效构建方法,并对其有效性进行了理论分析与实验验证。这些工作共同构成了对本文核心研究命题的系统性回应。

\mysection{局限性分析}
\label{sec:ch6-2-limitations-analysis}

尽管本文在深度模型高效构建的知识、结构与参数三个层面分别提出了创新性的方法并取得了一定的积极成果,但研究工作本身亦存在若干局限性,有待未来的工作进一步探索和完善。

首先,关于知识层面的 Prompt-Distiller 方法(第三章),其有效性在一定程度上依赖于高质量教师模型的可用性。该方法采用了双教师机制,同时需要一个经过提示微调的教师 ($\Theta_T$) 和一个原始的预训练教师 ($\Theta_{T'}$)。在某些场景下,获取或训练高质量的提示微调教师本身可能就需要一定的成本,而原始预训练教师的质量和适用性也直接影响蒸馏效果。此外,作为一种基于提示学习的方法,Prompt-Distiller 的性能可能受到提示模板设计的影响,尽管本文采用了公开模板,但寻找最优提示本身仍是一个挑战。同时,基于探针的对比学习策略的有效性,可能与探针自身的训练质量以及负样本的采样策略相关,需要仔细的调优。该方法主要在文本分类任务上进行了验证,其在更复杂的自然语言生成任务或跨模态场景下的适用性与有效性尚待进一步考察。

其次,针对结构层面的 BRIDGE 框架(第四章),其核心在于学习统一的架构表示与跨域映射。然而,学习一个能够有效捕捉不同异构搜索空间共性且与性能高度相关的通用潜在表示本身极具挑战性。当源域与目标域的架构范式差异过大时,表示学习的质量和后续映射的有效性可能下降。同时,学习跨域映射函数 $\mathcal{M}$ 的过程通常需要少量在目标域上评估过的架构样本作为锚点,这意味着 BRIDGE 并非完全零成本迁移,其效率提升程度与所需的目标域冷启动样本数量有关。此外,虽然进化序贯迁移优化策略旨在规避负迁移,但在某些源域与目标域极不匹配的情况下,迁移来的初始种群仍有可能误导搜索方向。最后,虽然 BRIDGE 显著降低了搜索成本,但架构表示学习本身(特别是基于 Transformer 的 VAE)也需要一定的预训练计算开销。

再者,对于参数层面的 KG-MFTO 范式(第五章),其在免训练模型融合方面展现了潜力,但也存在一些局限。该方法的可扩展性虽然优于基线,但当待融合模型的数量 $n$ 极大增加时,知识图谱的规模和 GNN 的训练复杂度会相应增长,动态维护和推理的成本可能成为新的瓶颈。同时,KG-MFTO 的有效性高度依赖于用于评估融合模型性能的函数 $F$ 的可靠性与稳定性。如果性能评估本身存在较大噪声或偏差,将直接影响知识图谱记录的准确性和 GNN 预测的质量,进而误导课程规划和求解器。此外,虽然知识引导的进化求解器(热启动 CMA-ES)旨在提升效率,但其最终效果仍受限于 GNN 预测的初始解质量和协同关系的准确度。最后,当前的参数融合研究(包括 KG-MFTO)大多假设待融合模型具有兼容的(例如,源自同一基础架构的)参数空间,对于融合来自完全不同架构体系的模型,该范式尚无法直接适用。

除上述针对具体方法的局限外,本文研究亦存在一些共性的待完善之处。本文提出的三项核心工作(Prompt-Distiller, BRIDGE, KG-MFTO)是在知识、结构、参数三个维度上独立展开的。我们并未深入探讨这些不同层面高效构建技术之间可能存在的协同作用或潜在冲突。例如,使用 BRIDGE 搜索到的高效架构是否更有利于后续的知识蒸馏或参数融合?反之,通过参数融合得到的模型是否能作为更好的教师模型?这些跨维度交互的问题有待未来研究。此外,本文的研究主要聚焦于提升模型构建的效率(数据、计算成本),对于知识迁移可能带来的其他影响,如模型的公平性、鲁棒性、可解释性等方面的变化,未做深入分析。知识迁移本身也可能传递源域数据或模型中存在的偏见,如何在追求效率的同时确保模型的可靠性与安全性,是需要持续关注的重要议题。

认识到这些局限性是推动研究不断前进的动力。未来的工作可以在克服这些局限、拓展方法适用性以及探索更深层次机理等方面展开,从而更全面地推进深度模型高效构建领域的发展。

\mysection{未来展望}
\label{sec:ch6-3-future-work}

本文围绕基于知识迁移的深度模型高效构建,在知识、结构与参数三个层面分别进行了探索并提出相应方法。尽管取得了一定的进展,但正如 6.3 节所分析,当前研究仍存在诸多局限性,同时也揭示了广阔的未来研究空间。立足于本文的工作基础与当前领域的挑战,未来值得进一步探索的方向主要包括以下几个方面:

首先,在知识层迁移方面,可以进一步深化少样本知识蒸馏的机制与应用场景。未来的工作可探索将 Prompt-Distiller 思想扩展至更复杂的任务,如少样本条件下的自然语言生成、序列标注或跨模态学习任务,研究如何在这些场景下有效传递教师模型的结构化知识或生成能力。此外,可以研究更自适应的知识迁移策略,例如自动化地学习最优的提示模板用于蒸馏,或者动态地调整不同知识源(如不同教师、不同层级特征)在蒸馏过程中的权重。探索更鲁棒的对比学习机制以应对教师模型可能存在的噪声或偏差,以及研究如何更深度地融合无监督或自监督信号来进一步减少对标注数据的依赖,也是有价值的方向。

其次,在结构层迁移方面,可以致力于发展更通用、更高效的跨异构架构知识迁移方法。针对 BRIDGE 框架的局限性,未来研究可探索更强大的神经网络架构表示学习模型,例如能够捕捉更细粒度结构信息或动态计算图特性的表示方法,以构建更具表达力的统一潜在空间。同时,研究更先进的跨域映射技术,如非线性映射、基于最优传输理论的分布对齐方法,或能够量化迁移不确定性的概率映射模型,有望提升异构迁移的精度与鲁棒性。将跨异构迁移思想扩展到多目标 NAS 场景(同时优化精度、延迟、能耗等),以及研究如何在几乎无需目标域标注样本的情况下实现零样本架构迁移,也是极具挑战性但意义重大的方向。此外,将 BRIDGE 框架应用于更广泛的架构类型(如图神经网络、循环神经网络)的迁移也是值得探索的扩展。

再者,在参数层迁移方面,可以持续优化免训练模型融合的可扩展性、鲁棒性与应用范围。随着预训练模型数量的持续增长,如何将 KG-MFTO 框架扩展以高效融合数十乃至上百个模型是一个重要的工程与算法挑战,可能需要更高效的知识图谱构建与推理机制,以及更可扩展的优化求解器。提升融合过程对性能评估噪声的鲁棒性,研究更精确的模型间协同/冲突关系的量化方法,以及探索融合结果的可解释性,对于增强该范式的可靠性至关重要。此外,突破当前参数融合大多局限于同构架构的限制,研究如何在存在架构差异的模型之间进行有效的参数(或功能)层面的融合,将极大地拓展该技术的应用边界。将 KG-MFTO 与 PEFT 技术(如 LoRA 模块的融合)相结合,探索参数高效调整与免训练融合的协同,也是一个有趣的方向。

最后,超越单个维度的独立研究,探索知识、结构、参数三个层面高效构建技术之间的协同与整合,是未来一个更宏大但也更复杂的方向。尽管本文强调了三项工作的独立性,但未来研究可以探讨它们之间潜在的相互促进作用。例如,由 \textsc{Bridge} 高效搜索出的紧凑架构,是否天然更适合通过 Prompt-Distiller 进行知识蒸馏?通过 KG-MFTO 融合得到的强力模型,是否能作为更好的教师模型或架构搜索的性能评估器?反之,知识蒸馏或参数融合技术能否用于压缩或优化 \textsc{Bridge} 搜索得到的架构?理解和利用这些跨维度交互,可能催生出更全局、更系统的高效模型构建新范式。此外,将模型构建的效率考量从单纯的数据、计算成本,扩展到更全面的可持续性指标(如碳排放、生命周期成本),并将公平性、鲁棒性、隐私保护等社会伦理因素纳入高效构建的优化目标中,也将是未来研究不可或缺的重要组成部分,推动“绿色 AI”与“可信 AI”的发展。

综上所述,深度模型的高效构建是一个充满挑战但也机遇无限的研究领域。本文的工作仅是其中的初步探索,期望能为未来的研究提供一些有益的启示,共同推动人工智能技术向着更高效、更普惠、更负责任的方向前进。

\end{document}