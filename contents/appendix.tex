\chapter[附\hskip\ccwd{}\hskip\ccwd{}录]{{\heiti\zihao{3}附\hskip\ccwd{}\hskip\ccwd{}录}}

\section[\hspace{-2pt}作者在攻读硕士学位期间的论文目录]{{\heiti\zihao{-3} \hspace{-8pt}作者在攻读硕士学位期间的论文目录}}

%下面是盲审标记\cs{secretize}的用法,记得去\textsf{main.tex}开启盲审开关看效果:

% \circled{1}已发表论文

% \begin{enumerate}
%     \item \textbf{\secretize{XU X}}, \secretize{LIU K}, DAI P, et al. Joint task offloading and resource optimization in NOMA-based vehicular edge computing: A game-theoretic DRL approach[J]. Journal of Systems Architecture, 2023, 134: 102780. 影响因子: 5.836(2021), 4.497(5年) (中科院SCI 2区,对应本文第三章)
% 	\item \textbf{\secretize{许新操}}, \secretize{刘凯}, 刘春晖, 等. 基于势博弈的车载边缘计算信道分配方法[J]. 电子学报, 2021,49(5): 851-860. (EI 索引,CCF T1类中文高质量科技期刊,对应本文第三章)
% 	\item \textbf{ \secretize{XU X}}, \secretize{LIU K}, XIAO K, et al. Vehicular fog computing enabled real-time collision warning via trajectory calibration[J]. Mobile Networks and Applications, 2020, 25(6): 2482-2494. 影响因子: 3.077(2021), 2.92(5年) (中科院SCI 3区,对应本文第五章)
% \end{enumerate}
{
\small
\setlength{\baselineskip}{20pt}
\begin{enumerate}[label={[\arabic*]}, leftmargin=*]
\item {\secretize{Yin G}}, \secretize{Huang S}, He T, et al. Mirrored EAST: An Efficient Detector for Automatic Vehicle Identification Number Detection in the Wild[J]. IEEE Transactions on Industrial Informatics, 2023. (中科院SCI一区)
\item {\secretize{Yin G}}, Wang Y, Zhang Y, et al. Adversarial Bidirectional Feature Generation for Generalized Zero-Shot Learning Under Unreliable Semantics[C]//Chinese Conference on Pattern Recognition and Computer Vision (PRCV). Cham: Springer Nature Switzerland, 2022: 639-654.(CCF-C)
\item {\secretize{Yin G}}, Huangfu L, \secretize{Huang S}, et al. Rethinking the Sample Relations for Few-Shot Classification[J]. Image and Vision Computing. (中科院SCI三区,返修中)
\end{enumerate}
}


\section[\hspace{-2pt}作者在攻读硕士学位期间参与的科研项目]{{\heiti\zihao{-3} \hspace{-8pt}作者在攻读硕士学位期间参与的科研项目}}

{
\small
\setlength{\baselineskip}{20pt}
\begin{enumerate}[label={[\arabic*]}, leftmargin=*]
\item 国家自然科学基金面上项目,少样本学习特征生成与鲁棒性关键技术研究
% (No. 62176030)
\item 重庆市自然科学基金面上项目,文本描述协同的双向生成式少样本学习研究
\end{enumerate}
}

% \newpage
% \section[\hspace{-2pt}学位论文数据集]{{\heiti\zihao{-3} \hspace{-8pt}学位论文数据集}}

% \begin{table}[h]
% \resizebox{\columnwidth}{!}{%
% \begin{tabular}{|cllcclclcl|}
% \hline
% \multicolumn{4}{|c|}{\heiti{关键词}}             & \multicolumn{2}{c|}{\heiti{密级}}   & \multicolumn{4}{c|}{\heiti{中图分类号}}                                    \\ \hline
% \multicolumn{4}{|c|}{\begin{tabular}[c]{@{}c@{}}车载信息物理融合系统;\\异构车联网; 车载边缘计算;\\资源优化; 多智能体深度强化学习\end{tabular}} & \multicolumn{2}{c|}{公开} & \multicolumn{4}{c|}{TP} \\ \hline
% \multicolumn{3}{|c|}{\heiti{学位授予单位名称}} & \multicolumn{3}{c|}{\heiti{学位授予单位代码}}    & \multicolumn{2}{c|}{\heiti{学位类别}}  & \multicolumn{2}{c|}{\heiti{学位级别}}        \\ \hline
% \multicolumn{3}{|c|}{\secretize{重庆大学}}     & \multicolumn{3}{c|}{\secretize{10611}}       & \multicolumn{2}{c|}{学术学位}  & \multicolumn{2}{c|}{博士}          \\ \hline
% \multicolumn{4}{|c|}{\heiti{论文题名}}            & \multicolumn{2}{c|}{\heiti{并列题名}} & \multicolumn{4}{c|}{\heiti{论文语种}}                                     \\ \hline
% \multicolumn{4}{|c|}{\begin{tabular}[c]{@{}c@{}}车载信息物理融合系统建模与优化关键技术研究\end{tabular}}               & \multicolumn{2}{c|}{无}   & \multicolumn{4}{c|}{中文} \\ \hline
% \multicolumn{3}{|c|}{\heiti{作者姓名}}     & \multicolumn{3}{c|}{\secretize{许新操}}         & \multicolumn{2}{c|}{\heiti{学号}}    & \multicolumn{2}{c|}{\secretize{20191401452}} \\ \hline
% \multicolumn{6}{|c|}{\heiti{培养单位名称}}                                      & \multicolumn{4}{c|}{\heiti{培养单位代码}}                                   \\ \hline
% \multicolumn{6}{|c|}{\secretize{重庆大学}}                                        & \multicolumn{4}{c|}{\secretize{10611}}                                    \\ \hline
% \multicolumn{3}{|c|}{\heiti{学科专业}}     & \multicolumn{3}{c|}{\heiti{研究方向}}        & \multicolumn{2}{c|}{\heiti{学制}}    & \multicolumn{2}{c|}{\heiti{学位授予年}}       \\ \hline
% \multicolumn{3}{|c|}{计算机科学与技术} & \multicolumn{3}{c|}{车联网}         & \multicolumn{2}{c|}{4年}     & \multicolumn{2}{c|}{\secretize{2023年}}        \\ \hline
% \multicolumn{3}{|c|}{\heiti{论文提交日期}}   & \multicolumn{3}{c|}{\secretize{2023年6月}}     & \multicolumn{2}{c|}{\heiti{论文总页数}} & \multicolumn{2}{c|}{\pageref{LastPage}}         \\ \hline
% \multicolumn{3}{|c|}{\heiti{导师姓名}}     & \multicolumn{3}{c|}{\secretize{刘凯}}          & \multicolumn{2}{c|}{\heiti{职称}}    & \multicolumn{2}{c|}{教授}          \\ \hline
% \multicolumn{6}{|c|}{\heiti{答辩委员会主席}}                                     & \multicolumn{4}{c|}{\secretize{雒江涛}}                                      \\ \hline
% \multicolumn{10}{|c|}{\heiti{\begin{tabular}[c]{@{}c@{}} 电子版论文提交格式\\ 文本(\checkmark) 图像() 视频()音频()多媒体()其他()\end{tabular}}}                              \\ \hline
% \end{tabular}%
% }
% \end{table}