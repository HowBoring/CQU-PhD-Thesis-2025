\chapter[致\hskip\ccwd{}\hskip\ccwd{}谢]{{\heiti\zihao{3}致\hskip\ccwd{}\hskip\ccwd{}谢}}

% 这里用盲审环境包裹致谢,在开启盲审开关时,环境内部的内容不予渲染。
\begin{secretizeEnv}

提笔之际,已到了在重庆生活学习的第六个年头,而我的博士研究生学习阶段也可以说算是告一段落。回想读博期间一路走来,其中有欣喜,也有难过;有深深的孤独,也有现在的恋恋不舍。如今终于到了要道别的时候,所以想借此机会给每一个支持和帮助我的人们好好说一声感谢与有缘再见。

首先,我要衷心感谢我的导师刘凯教授。您是我学术道路上的引路人,您的悉心指导对我产生了巨大影响。您的专业知识、学术见解和研究激情都激发了我不断超越自我的动力。您耐心地解答我的问题,指导我的实验,并对我的论文提出宝贵的建议。您对我的信任和鼓励使我更加自信地迈向学术领域的新阶段。我将永远铭记您对我的慷慨付出和关心。

其次,我要感谢西南交通大学戴朋林老师、重庆邮电大学张浩老师、重庆师范大学肖颗老师、重庆大学国家卓越工程师学院李楚照老师,以及实验室廖成武、金飞宇、任华玲、刘春晖、晏国志、胡峻菠、钟成亮、吴峻源等同学在本学位论文撰写和校对过程中提供的宝贵意见与无私帮助。

再次,我要感谢我的母亲刘菁女士。您生育抚养了我,感谢您对我的无私包容与关爱支持,如果可以,我希望把这篇论文献给您。您是我所见过最坚强的人,都说\qthis{为母则刚},但我也希望能有一天,您能放下心中的重担,为自己好好生活。

此外,我也要感谢实验室的师弟师妹们。在毕业之际,我们一起欢聚于重庆大学国家卓越工程师学院,一起度过了许多个日夜,也为我带来了难忘的回忆。

最后,我要特别感谢答辩主席中国科学院重庆绿色智能技术研究院尚明生教授和所有委员重庆大学郭松涛教授、西北工业大学王柱教授、重庆邮电大学高陈强教授、重庆大学古富强教授的仔细审查和评估。感谢你们在繁忙的工作中抽出时间来对我的研究进行评价,并给予我宝贵的意见和建议。同时,我还要感谢论文评审专家们,你们在匿名评审的过程中,以专业、客观的态度审查了我的论文。你们对我的研究提出的批评和建议,帮助我更好地认识到研究的不足之处,并鼓励我在今后的学术探索中不断进步和改进。衷心感谢各位论文评审专家与答辩委员专家的辛勤工作和付出。

感谢你们陪伴我度过漫长岁月,世界因你们更美好。
\vfill
\begin{flushright}
{\stxingkai \Large 许新操} \hspace*{3.5em}
\\  \hspace*{\fill} \\
{二〇二三年五月\hspace*{1em}于重庆}
\end{flushright}
\end{secretizeEnv}