% !TeX encoding = UTF-8
%% \textbf{重庆大学}通用毕业论文\LaTeXe{}模板
%%% 使用前请先阅读使用文档和用户协议,内有详细介绍。Happy Texing! :)
%% =======================================================
\documentclass[
	type=doctor,
	bilinguallist=apart, 
	printmode=twoside,
	bilinguallist=off,
	blindtrail=false,
	draft=true,
]{cquthesis}%
% 可用选项:
% type=[bachelor|master|doctor],      % 必选,毕业论文类型,以下项目不填时为默认
% liberalformat,                      % 可选,仅适用本科生,使用文学类论文标题格式,默认未打开
% proffesionalmaster=[true|false],    % 可选,仅适用研究生,是(true)否(false)专业硕士,默认为否
% printmode=[oneside|twoside|auto],	  % 可选,论文打印方式,默认采用auto按页数要求自动判定
% openany,|openright,                 % 可选,双面打印时每章的第一页仅右页开启,默认右页开启(openright)
% bilinguallist=[off|combined|apart], % 可选,图录表录等分别按双语题注混编(combined),分开编录(apart),默认关(off)
% blindtrail=[true|false],            % 可选,盲审模式,开启后封面姓名和致谢部分会隐藏,详情请参阅用户文档,默认关
% draft,                              % 写作期间可选,不渲染图片,关闭外围功能,加快预览速度,默认未开启

% 请在cquthesis.sty文件中定义其他会用到的宏包和自己的变量
% 这样可以防止main.tex太过臃肿。
% \usepackage[subpreambles=true]{standalone}
\usepackage{subfiles}

\usepackage{cquthesis}
\usepackage{newtxmath}

% \setCJKfamilyfont{hyxrwsong}[
% 	UprightFont = {HYXinRenWenSong 55W},
% 	BoldFont = {HYXinRenWenSong 65W},
% ]{HYXinRenWenSong}
% \renewcommand\songti{\CJKfamily{hyxrwsong}\selectfont}
% \setCJKmainfont[
% 	UprightFont = {HYXinRenWenSong 55W},
% 	BoldFont = {HYXinRenWenSong 65W},
% 	ItalicFont = {HYKaiTiS.ttf},
% ]{HYXinRenWenSong}
% \defaultfontfeatures{hyxrwsong}
\setCJKmainfont{HYXinRenWenSong}[
  Path = {/usr/share/fonts/win-fonts/},
  UprightFont = HYXinRenWenSong55W,
  BoldFont = HYXinRenWenSong65W,
  ItalicFont = HYKaiTiS,
  Extension = .ttf
]

% 定义所有的图片文件在 figures 子目录下
\graphicspath{{figures/}}

% 定义数字圆
% \usepackage{tikz}
% \newcommand*\circled[1]{\tikz[baseline=(char.base)]{
%             \node[shape=circle,draw,inner sep=1pt] (char) {\small #1};}}
\usepackage{tikz}
\newcommand*\circled[1]{\tikz[baseline=(char.base)]{
    \node[shape=circle,draw,inner sep=1pt,
          % Check if current font is bold, then adjust line width and font
          /utils/exec=\ifx\f@series\bfseries\pgfextra{\pgfkeyssetvalue{/tikz/line width}{1pt}\global\let\tikz@textfont=\bfseries}\fi
         ] (char) {#1};
}}

% 自定义章节命令(含 \hspace 控制)
\newcommand{\mychapter}[1][]{ \chapter[\hspace{0pt}#1]{{\heiti\zihao{3}\hspace{0pt}#1}} }
\newcommand{\mysection}[1][]{ \section[\hspace{-2pt}#1]{{\heiti\zihao{-3}\hspace{-8#1pt}}} }
\newcommand{\mysubsection}[1][]{ \subsection[\hspace{-2pt}#1]{{\heiti\zihao{4}\hspace{-8#1pt}}} }

\begin{document}


\cqusetup{
%	************	注意	************
%	* 1. \cqusetup{}中不能出现全空的行,如果需要全空行请在行首注释
%	* 2. 不需要的配置信息可以放心地坐视不理、留空、删除或注释(都不会有影响)
%	*
%	********************************
% ===================
%	论文的中英文题目
% ===================
  ctitle = {基于多元关系建模的少样本分类算法研究},
  etitle = {Research on Few-Shot Classification Algorithms Based on Multivariate Relationship Modeling},
% ===================
% 作者部分的信息
% \secretize{}为盲审标记点,在打开盲审开关时内容会自动被替换为***输出,盲审开关默认关闭
% ===================
  cauthor = \secretize{侯博宇},	% 你的姓名,以下每项都以英文逗号结束
  eauthor = \secretize{Boyu~Hou},	% 姓名拼音,~代表不会断行的空格
  studentid = \secretize{20211401002},	% 仅本科生,学号
  csupervisor = \secretize{冯~~~亮~~~~~教授},	% 导师的姓名
  esupervisor = \secretize{{Prof.~Liang Feng}},	% 导师的姓名拼音
  cassistsupervisor = \secretize{}, % 本科生可选,助理指导教师姓名,不用时请留空为{}
  cextrasupervisor = \secretize{}, % 本科生可选,校外指导教师姓名,不用时请留空为{}
  eassistsupervisor = \secretize{}, % 本科生可选,助理指导教师或/和校外指导教师姓名拼音,不用时请留空为{}
  cpsupervisor = \secretize{}, % 仅专硕,兼职导师姓名
  epsupervisor = \secretize{},	% 仅专硕,兼职导师姓名拼音
  cclass = \secretize{\rmfamily{2025}\heiti{年}\rmfamily{6}\heiti{月}},	% 博士生和学硕填学科门类,学硕填学科类型
  research_direction = \zihao{3}{工学},
  edgree = {},	% 专硕填Professional Degree,其他按实情填写
% % 提示:如果内容太长,可以用\zihao{}命令控制字号,作用范围:{}内
  cmajor = 工~~~学,	% 专硕不需填,填写专业名称
  emajor = , % % 专硕不需填,填写专业英文名称
  cmajora = \zihao{3}{计算机科学与技术},	% 专硕不需填,填写专业名称
  cmajorb = \zihao{3}{人工智能},
  cmajorc = \secretize{},
  % cmajord = 2024年6月,
% ===================
% 底部的学院名称和日期
% ===================
  cdepartment = 计算机学院,	%学院名称
  edepartment = College of Computer Science, %学院英文名称
% ===================
% 封面的日期可以自动生成(注释掉时),也可以解除注释手动指定,例如:二〇一六年五月
% ===================
%	mycdate = {2023年6月},
%	myedate = {June 2023},
}% End of \cqusetup
% ===================
%
% 论文的摘要
%
% ===================
\begin{cabstract}	% 中文摘要
近年来,深度学习技术已在诸多图像分类任务上取得了显著成就,但这些任务的成功往往依赖于海量的标注数据,在数据匮乏的情况下,很多深度学习模型便无法精确识别物体类别。为了提升深度学习模型在数据匮乏情况下的效果,旨在模拟人类识别物体过程的少样本分类(Few-Shot Classification,简称FSC)被提出并取得了一定进展。在少样本分类任务中,如何将模型在大量数据上学习到的知识迁移到新的类别是解决问题的关键。虽然基类数据与新类数据的类别不同,但其数据间的多种关系是具有相似性和关联性的。通过在基类数据上对这些关系进行建模,可以更好地理解与挖掘数据间的内在联系,从而将在基类上学习到的知识迁移至新类。基于此,本文以数据的多元关系为切入点,对多粒度样本关系和语义-视觉多空间关系进行建模,以推动少样本分类问题的研究进展。本文主要工作如下:

(1)针对少样本分类模型特征提取能力不足的问题,本文开展基于多粒度样本关系建模的少样本分类研究,提出了多粒度样本关系对比学习(Multi-Grained Sample Relation Contrastive Learning,简称MGSRCL)模型。该模型将样本关系划分为三种类型:样本内关系、类内关系和类间关系,并使用变换一致性学习约束样本内关系,使用类对比学习约束类内关系与类间关系,对多种粒度的样本关系进行充分挖掘与细致建模。在多个基准数据集上的大量实验证明,MGSRCL方法通过建模多粒度样本关系提升了模型的特征提取能力与泛化能力,有效提高了少样本分类准确率,并为其他两阶段方法提供了一个优质的预训练模型。

(2)针对仅根据少量样本的视觉特征无法捕获类别代表性特征的问题,本文开展基于语义-视觉多空间关系建模的少样本分类研究,提出了语义-视觉多空间映射适配(Semantic-Visual Multi-Space Mapping Adapter,简称SVMSMA)模型。该模型引入语义信息作为视觉信息的补充,通过语义-视觉多空间映射网络将语义特征映射到视觉空间,并使用简单有效的跨模态分类和跨模态特征对齐策略对语义特征与视觉特征进行建模。大量实验证明,SVMSMA模型能够有效建立语义信息与视觉信息的联系,丰富了样本特征的信息来源,从而能够利用语义信息提升样本特征的多样性,增强模型对新类别的适应能力和泛化能力,并在MGSRCL的基础上进一步提升了少样本分类准确率。

\end{cabstract}
% 中文关键词,请使用英文逗号分隔:
\ckeywords{少样本分类;关系建模;对比学习;语义信息表示}

\begin{eabstract}	% 英文摘要

In recent years, deep learning technology has made remarkable achievements in image classification tasks. But the success of these tasks often depends on vast amounts of labeled data, and in the absence of data, many deep learning models are unable to accurately identify object categories. In order to improve the effectiveness of deep learning models in the case of data scarcity, Few-Shot Classification (FSC), which aims to simulate the object recognition process of human, has been proposed and made some progress. In FSC, how to transfer the knowledge learned by the model on a large amount of data to new categories is the key to solve the problem. Although the categories of the base classes and the novel classes are different, the relationship between the data is similar and relevant. By modeling these relationships on the base classes, the internal relationship between the data can be better understood and mined, so that the knowledge learned on the base classes can be transferred to the novel classes. Based on this, this paper takes the multivariate relationship of data as the starting point to model the multi-grained sample relation and semantic-visual multi-space relation, so as to promote the research progress of FSC. The main works of this paper are as follows:

(1) Aiming at the lack of feature extraction ability of FSC model, this paper carries out a study of FSC based on multi-grained sample relationship modeling, and proposes a Multi-Grained Sample Relation Contrastive Learning (MGSRCL) model. The model divides the sample relations into three types: intra-sample relation, intra-class relation and inter-class relation, and uses transformation consistency learning to constrain intra-sample relation, uses class contrastive learning to constrain intra-class relation and inter-class relation, so as to fully mine and carefully model the sample relations of various granularity. Extensive experiments on multiple benchmark datasets prove that MGSRCL can improve the feature extraction ability and generalization ability of the model by modeling multi-grained sample relations, effectively improve the accuracy of FSC, and provide an excellent pre-training model for other two-stage methods.

(2) In view of the problem that the category representative features cannot be captured only based on the visual features of a small number of samples, this paper carries out a research on FSC based on semantic-visual multi-space relationship modeling, and proposes a Semantic-Visual Multi-Space Mapping Adapter (SVMSMA) model. The model introduces semantic information as a supplement to visual information, maps semantic features to visual space through semantic-visual multi-space mapping network, and uses simple and effective cross-modal classification and cross-modal feature alignment strategies to model semantic features and visual features. Extensive experiments have proved that SVMSMA model can effectively establish the connection between semantic information and visual information, enrich the information sources of sample features, and thus improve the diversity of sample features by using semantic information, enhance the model's adaptability and generalization ability to new categories, and further improve the accuracy of FSC on the basis of MGSRCL.
 
\end{eabstract}
% 英文关键词,请使用英文逗号分隔,关键词内可以空格:
\ekeywords{Few-Shot Classification, Relationship Modeling, Contrastive Learning, Semantic Information Representation
}

% 封面和摘要配置完成
% 封面部分
\makecover

\frontmatter %%%前置部分(封面后绪论前)
% \cquauthpage[contents/cover1.pdf]
% \cquauthpage[contents/cover2.pdf]
%\cquauthpage[contents/cover3.pdf]
%\cquauthpage[contents/cover4.pdf]

%% 原创声明和授权说明书,可选:用扫描页替换
%\cquauthpage[authscan.pdf]
%\cquauthpage

% 摘要
\makeabstract

%% 目录,注意需要多次编译才能更新
\setlength{\cftbeforetoctitleskip}{0pt}
\setlength{\cftaftertoctitleskip}{20pt}
\tableofcontents


% \setlength{\cftbeforelottitleskip}{0pt}
% \setlength{\cftafterlottitleskip}{20pt}
%% 插图索引,可选,如不用可注释掉
% \renewcommand*{\listfigurename}{图目录}
% \clearpage
% \phantomsection
% \addcontentsline{toc}{chapter}{图目录}
% \listoffigures
%\listoffiguresEN
% \setlength{\cftbeforelottitleskip}{0pt}
% \setlength{\cftafterlottitleskip}{20pt}
%% 表格索引,可选
% \renewcommand*{\listtablename}{表目录}
% \clearpage
% \phantomsection
% \addcontentsline{toc}{chapter}{表目录}
% \listoftables
%\listoftablesEN
%% 公式索引,可选
%\listofequations
%\listofequationsEN
%% 符号对照表,可选
% \clearpage
% \phantomsection
% \addcontentsline{toc}{chapter}{主要符号对照表}
% \input{contents/denotation}
% %% 缩略语对照表,可选
% \clearpage
% \phantomsection
% \addcontentsline{toc}{chapter}{缩略语对照表}
% \input{contents/abbreviate}

\mainmatter %%% 主体部分(绪论开始,结论为止)
%* 子文件的多少和内容由你决定(最好以章为单位),基本原则是提速预览、脉络清晰、管理容易。

% 设置字号为小四
\renewcommand{\normalsize}{\fontsize{12pt}{20pt}\selectfont}
% 设置小四正文行间距为 20 磅
\setstretch{1.312}

\subfile{contents/introduction}

\backmatter %%% 后置部分(致谢、参考文献、附录等)

%% 参考文献
% 顺序编码制:cqunumerical
% 注意:至少需要引用一篇参考文献,否则下面两行会引起编译错误。
% \bibliographystyle{cqunumerical}
\bibliographystyle{gbt7714-numerical_new}
\bibliography{ref/refs}


%% 附录(按ABC...分节,证明、推导、程序、个人简历等)
\appendix
% \chapter[附\hskip\ccwd{}\hskip\ccwd{}录]{{\heiti\zihao{3}附\hskip\ccwd{}\hskip\ccwd{}录}}

\section[\hspace{-2pt}作者在攻读硕士学位期间的论文目录]{{\heiti\zihao{-3} \hspace{-8pt}作者在攻读硕士学位期间的论文目录}}

%下面是盲审标记\cs{secretize}的用法,记得去\textsf{main.tex}开启盲审开关看效果:

% \circled{1}已发表论文

% \begin{enumerate}
%     \item \textbf{\secretize{XU X}}, \secretize{LIU K}, DAI P, et al. Joint task offloading and resource optimization in NOMA-based vehicular edge computing: A game-theoretic DRL approach[J]. Journal of Systems Architecture, 2023, 134: 102780. 影响因子: 5.836(2021), 4.497(5年) (中科院SCI 2区,对应本文第三章)
% 	\item \textbf{\secretize{许新操}}, \secretize{刘凯}, 刘春晖, 等. 基于势博弈的车载边缘计算信道分配方法[J]. 电子学报, 2021,49(5): 851-860. (EI 索引,CCF T1类中文高质量科技期刊,对应本文第三章)
% 	\item \textbf{ \secretize{XU X}}, \secretize{LIU K}, XIAO K, et al. Vehicular fog computing enabled real-time collision warning via trajectory calibration[J]. Mobile Networks and Applications, 2020, 25(6): 2482-2494. 影响因子: 3.077(2021), 2.92(5年) (中科院SCI 3区,对应本文第五章)
% \end{enumerate}
{
\small
\setlength{\baselineskip}{20pt}
\begin{enumerate}[label={[\arabic*]}, leftmargin=*]
\item {\secretize{Yin G}}, \secretize{Huang S}, He T, et al. Mirrored EAST: An Efficient Detector for Automatic Vehicle Identification Number Detection in the Wild[J]. IEEE Transactions on Industrial Informatics, 2023. (中科院SCI一区)
\item {\secretize{Yin G}}, Wang Y, Zhang Y, et al. Adversarial Bidirectional Feature Generation for Generalized Zero-Shot Learning Under Unreliable Semantics[C]//Chinese Conference on Pattern Recognition and Computer Vision (PRCV). Cham: Springer Nature Switzerland, 2022: 639-654.(CCF-C)
\item {\secretize{Yin G}}, Huangfu L, \secretize{Huang S}, et al. Rethinking the Sample Relations for Few-Shot Classification[J]. Image and Vision Computing. (中科院SCI三区,返修中)
\end{enumerate}
}


\section[\hspace{-2pt}作者在攻读硕士学位期间参与的科研项目]{{\heiti\zihao{-3} \hspace{-8pt}作者在攻读硕士学位期间参与的科研项目}}

{
\small
\setlength{\baselineskip}{20pt}
\begin{enumerate}[label={[\arabic*]}, leftmargin=*]
\item 国家自然科学基金面上项目,少样本学习特征生成与鲁棒性关键技术研究
% (No. 62176030)
\item 重庆市自然科学基金面上项目,文本描述协同的双向生成式少样本学习研究
\end{enumerate}
}

% \newpage
% \section[\hspace{-2pt}学位论文数据集]{{\heiti\zihao{-3} \hspace{-8pt}学位论文数据集}}

% \begin{table}[h]
% \resizebox{\columnwidth}{!}{%
% \begin{tabular}{|cllcclclcl|}
% \hline
% \multicolumn{4}{|c|}{\heiti{关键词}}             & \multicolumn{2}{c|}{\heiti{密级}}   & \multicolumn{4}{c|}{\heiti{中图分类号}}                                    \\ \hline
% \multicolumn{4}{|c|}{\begin{tabular}[c]{@{}c@{}}车载信息物理融合系统;\\异构车联网; 车载边缘计算;\\资源优化; 多智能体深度强化学习\end{tabular}} & \multicolumn{2}{c|}{公开} & \multicolumn{4}{c|}{TP} \\ \hline
% \multicolumn{3}{|c|}{\heiti{学位授予单位名称}} & \multicolumn{3}{c|}{\heiti{学位授予单位代码}}    & \multicolumn{2}{c|}{\heiti{学位类别}}  & \multicolumn{2}{c|}{\heiti{学位级别}}        \\ \hline
% \multicolumn{3}{|c|}{\secretize{重庆大学}}     & \multicolumn{3}{c|}{\secretize{10611}}       & \multicolumn{2}{c|}{学术学位}  & \multicolumn{2}{c|}{博士}          \\ \hline
% \multicolumn{4}{|c|}{\heiti{论文题名}}            & \multicolumn{2}{c|}{\heiti{并列题名}} & \multicolumn{4}{c|}{\heiti{论文语种}}                                     \\ \hline
% \multicolumn{4}{|c|}{\begin{tabular}[c]{@{}c@{}}车载信息物理融合系统建模与优化关键技术研究\end{tabular}}               & \multicolumn{2}{c|}{无}   & \multicolumn{4}{c|}{中文} \\ \hline
% \multicolumn{3}{|c|}{\heiti{作者姓名}}     & \multicolumn{3}{c|}{\secretize{许新操}}         & \multicolumn{2}{c|}{\heiti{学号}}    & \multicolumn{2}{c|}{\secretize{20191401452}} \\ \hline
% \multicolumn{6}{|c|}{\heiti{培养单位名称}}                                      & \multicolumn{4}{c|}{\heiti{培养单位代码}}                                   \\ \hline
% \multicolumn{6}{|c|}{\secretize{重庆大学}}                                        & \multicolumn{4}{c|}{\secretize{10611}}                                    \\ \hline
% \multicolumn{3}{|c|}{\heiti{学科专业}}     & \multicolumn{3}{c|}{\heiti{研究方向}}        & \multicolumn{2}{c|}{\heiti{学制}}    & \multicolumn{2}{c|}{\heiti{学位授予年}}       \\ \hline
% \multicolumn{3}{|c|}{计算机科学与技术} & \multicolumn{3}{c|}{车联网}         & \multicolumn{2}{c|}{4年}     & \multicolumn{2}{c|}{\secretize{2023年}}        \\ \hline
% \multicolumn{3}{|c|}{\heiti{论文提交日期}}   & \multicolumn{3}{c|}{\secretize{2023年6月}}     & \multicolumn{2}{c|}{\heiti{论文总页数}} & \multicolumn{2}{c|}{\pageref{LastPage}}         \\ \hline
% \multicolumn{3}{|c|}{\heiti{导师姓名}}     & \multicolumn{3}{c|}{\secretize{刘凯}}          & \multicolumn{2}{c|}{\heiti{职称}}    & \multicolumn{2}{c|}{教授}          \\ \hline
% \multicolumn{6}{|c|}{\heiti{答辩委员会主席}}                                     & \multicolumn{4}{c|}{\secretize{雒江涛}}                                      \\ \hline
% \multicolumn{10}{|c|}{\heiti{\begin{tabular}[c]{@{}c@{}} 电子版论文提交格式\\ 文本(\checkmark) 图像() 视频()音频()多媒体()其他()\end{tabular}}}                              \\ \hline
% \end{tabular}%
% }
% \end{table}


%% 致谢
% \chapter[致\hskip\ccwd{}\hskip\ccwd{}谢]{{\heiti\zihao{3}致\hskip\ccwd{}\hskip\ccwd{}谢}}

% 这里用盲审环境包裹致谢,在开启盲审开关时,环境内部的内容不予渲染。
\begin{secretizeEnv}

提笔之际,已到了在重庆生活学习的第六个年头,而我的博士研究生学习阶段也可以说算是告一段落。回想读博期间一路走来,其中有欣喜,也有难过;有深深的孤独,也有现在的恋恋不舍。如今终于到了要道别的时候,所以想借此机会给每一个支持和帮助我的人们好好说一声感谢与有缘再见。

首先,我要衷心感谢我的导师刘凯教授。您是我学术道路上的引路人,您的悉心指导对我产生了巨大影响。您的专业知识、学术见解和研究激情都激发了我不断超越自我的动力。您耐心地解答我的问题,指导我的实验,并对我的论文提出宝贵的建议。您对我的信任和鼓励使我更加自信地迈向学术领域的新阶段。我将永远铭记您对我的慷慨付出和关心。

其次,我要感谢西南交通大学戴朋林老师、重庆邮电大学张浩老师、重庆师范大学肖颗老师、重庆大学国家卓越工程师学院李楚照老师,以及实验室廖成武、金飞宇、任华玲、刘春晖、晏国志、胡峻菠、钟成亮、吴峻源等同学在本学位论文撰写和校对过程中提供的宝贵意见与无私帮助。

再次,我要感谢我的母亲刘菁女士。您生育抚养了我,感谢您对我的无私包容与关爱支持,如果可以,我希望把这篇论文献给您。您是我所见过最坚强的人,都说\qthis{为母则刚},但我也希望能有一天,您能放下心中的重担,为自己好好生活。

此外,我也要感谢实验室的师弟师妹们。在毕业之际,我们一起欢聚于重庆大学国家卓越工程师学院,一起度过了许多个日夜,也为我带来了难忘的回忆。

最后,我要特别感谢答辩主席中国科学院重庆绿色智能技术研究院尚明生教授和所有委员重庆大学郭松涛教授、西北工业大学王柱教授、重庆邮电大学高陈强教授、重庆大学古富强教授的仔细审查和评估。感谢你们在繁忙的工作中抽出时间来对我的研究进行评价,并给予我宝贵的意见和建议。同时,我还要感谢论文评审专家们,你们在匿名评审的过程中,以专业、客观的态度审查了我的论文。你们对我的研究提出的批评和建议,帮助我更好地认识到研究的不足之处,并鼓励我在今后的学术探索中不断进步和改进。衷心感谢各位论文评审专家与答辩委员专家的辛勤工作和付出。

感谢你们陪伴我度过漫长岁月,世界因你们更美好。
\vfill
\begin{flushright}
{\stxingkai \Large 许新操} \hspace*{3.5em}
\\  \hspace*{\fill} \\
{二〇二三年五月\hspace*{1em}于重庆}
\end{flushright}
\end{secretizeEnv}

\end{document}