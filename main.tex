% !TeX encoding = UTF-8
%% \textbf{重庆大学}通用毕业论文\LaTeXe{}模板
%%% 使用前请先阅读使用文档和用户协议,内有详细介绍。Happy Texing! :)
%% =======================================================
\documentclass[
	type=doctor,
	bilinguallist=apart,
	printmode=twoside,
	bilinguallist=off,
	blindtrail=false,
	draft=false,
]{cquthesis}%
% 可用选项:
% type=[bachelor|master|doctor],      % 必选,毕业论文类型,以下项目不填时为默认
% liberalformat,                      % 可选,仅适用本科生,使用文学类论文标题格式,默认未打开
% proffesionalmaster=[true|false],    % 可选,仅适用研究生,是(true)否(false)专业硕士,默认为否
% printmode=[oneside|twoside|auto],	  % 可选,论文打印方式,默认采用auto按页数要求自动判定
% openany,|openright,                 % 可选,双面打印时每章的第一页仅右页开启,默认右页开启(openright)
% bilinguallist=[off|combined|apart], % 可选,图录表录等分别按双语题注混编(combined),分开编录(apart),默认关(off)
% blindtrail=[true|false],            % 可选,盲审模式,开启后封面姓名和致谢部分会隐藏,详情请参阅用户文档,默认关
% draft,                              % 写作期间可选,不渲染图片,关闭外围功能,加快预览速度,默认未开启

% 请在cquthesis.sty文件中定义其他会用到的宏包和自己的变量
% 这样可以防止main.tex太过臃肿。
% \usepackage[subpreambles=true]{standalone}
\usepackage{subfiles}
\usepackage{cquthesis}
\usepackage{bm}
\usepackage{nicematrix}
\usepackage{rotating}
\usepackage{longtable}
\usepackage{xltabular}

% \setCJKmainfont{tongti}[
%   Path = {/usr/share/fonts/win-fonts/},
%   UprightFont = {方正新书宋_GB18030},
%   AutoFakeBold = 3,
%   % BoldFont = HYXinRenWenSong65W,
%   ItalicFont = HYKaiTiS,
%   Extension = .ttf
% ]

\usepackage{mathtools}
\usepackage{newtxtext,newtxmath}
\usepackage[cal=pxtx]{mathalpha}

% 定义所有的图片文件在 figures 子目录下
\graphicspath{{figures/}}

\newcolumntype{L}{>{\raggedright\arraybackslash}X}

% 定义数字圆
% \usepackage{tikz}
% \newcommand*\circled[1]{\tikz[baseline=(char.base)]{
%             \node[shape=circle,draw,inner sep=1pt] (char) {\small #1};}}
\usepackage{tikz}
\newcommand*\circled[1]{\tikz[baseline=(char.base)]{
    \node[shape=circle,draw,inner sep=1pt,
          % Check if current font is bold, then adjust line width and font
          /utils/exec=\ifx\f@series\bfseries\pgfextra{\pgfkeyssetvalue{/tikz/line width}{1pt}\global\let\tikz@textfont=\bfseries}\fi
         ] (char) {#1};
}}

\providecommand{\mychapter}[2][]{%
  \if\relax\detokenize{#1}\relax
    \chapter[\hspace{0pt}#2]{\heiti\zihao{3}\hspace{0pt}#2}%
  \else
    \chapter[\hspace{0pt}#1]{\heiti\zihao{3}\hspace{0pt}#2}%
  \fi
}

\providecommand{\mysection}[2][]{%
  \if\relax\detokenize{#1}\relax
    \section[\hspace{-2pt}#2]{\heiti\zihao{-3}\hspace{-8pt}#2}%
  \else
    \section[\hspace{-2pt}#1]{\heiti\zihao{-3}\hspace{-8pt}#2}%
  \fi
}

\providecommand{\mysubsection}[2][]{%
  \if\relax\detokenize{#1}\relax
    \subsection[\hspace{-2pt}#2]{\heiti\zihao{4}\hspace{-8pt}#2}%
  \else
    \subsection[\hspace{-2pt}#1]{\heiti\zihao{4}\hspace{-8pt}#2}%
  \fi
}

\providecommand{\mysubsubsection}[1]{%
	\refstepcounter{subsubsection}%
	\par\vspace{0.5em}%
	\textbf{(\arabic{subsubsection})#1}\par\vspace{0.2em}%
}

\setlist[enumerate]{
  label={(\arabic*)},
  labelindent=2\ccwd,
  leftmargin=!}

\setlist[itemize]{
  labelindent=2\ccwd,
  leftmargin=!}

\begin{document}


\cqusetup{
	%	************	注意	************
	%	* 1. \cqusetup{}中不能出现全空的行,如果需要全空行请在行首注释
	%	* 2. 不需要的配置信息可以放心地坐视不理、留空、删除或注释(都不会有影响)
	%	*
	%	********************************
	% ===================
	%	论文的中英文题目
	% ===================
	ctitle = {基于知识迁移的深度模型高效构建研究},
	etitle = {Efficient Deep Model Construction via Knowledge Transfer},
	% ===================
	% 作者部分的信息
	% \secretize{}为盲审标记点,在打开盲审开关时内容会自动被替换为***输出,盲审开关默认关闭
	% ===================
	cauthor = \secretize{侯博宇},	% 你的姓名,以下每项都以英文逗号结束
	eauthor = \secretize{Boyu~Hou},	% 姓名拼音,~代表不会断行的空格
	studentid = \secretize{20211401002},	% 仅本科生,学号
	csupervisor = \secretize{冯~~~亮~~~~~教授},	% 导师的姓名
	esupervisor = \secretize{{Prof.~Liang Feng}},	% 导师的姓名拼音
	cassistsupervisor = \secretize{}, % 本科生可选,助理指导教师姓名,不用时请留空为{}
	cextrasupervisor = \secretize{}, % 本科生可选,校外指导教师姓名,不用时请留空为{}
	eassistsupervisor = \secretize{}, % 本科生可选,助理指导教师或/和校外指导教师姓名拼音,不用时请留空为{}
	cpsupervisor = \secretize{}, % 仅专硕,兼职导师姓名
	epsupervisor = \secretize{},	% 仅专硕,兼职导师姓名拼音
	cclass = \secretize{\rmfamily{2025}\heiti{年}\rmfamily{12}\heiti{月}},	% 博士生和学硕填学科门类,学硕填学科类型
	edgree = {},	% 专硕填Professional Degree,其他按实情填写
	% % 提示:如果内容太长,可以用\zihao{}命令控制字号,作用范围:{}内
	cmajor = {工~~~~学},	% 专硕不需填,填写专业名称
	emajor = {Computer Science and Technology}, % % 专硕不需填,填写专业英文名称
	cmajora = {计算机科学与技术},	% 专硕不需填,填写专业名称
	cmajorb = {深度学习},
	% cmajorc = \secretize{},
	% cmajord = 2024年6月,
	% ===================
	% 底部的学院名称和日期
	% ===================
	cdepartment = 计算机学院,	%学院名称
	edepartment = College of Computer Science, %学院英文名称
	% ===================
	% 封面的日期可以自动生成(注释掉时),也可以解除注释手动指定,例如:二〇一六年五月
	% ===================
	mycdate = {2025年12月},
	myedate = {December 2025},
}% End of \cqusetup
% ===================
%
% 论文的摘要
%
% ===================
\begin{cabstract}	% 中文摘要
	深度学习在众多领域取得了显著成就,但其模型性能的提升往往伴随着数据、算力与人力成本的急剧增长,形成了严峻的“规模-效率”矛盾,制约了技术的广泛应用与可持续发展。为应对这一挑战,本文以“知识迁移”作为核心指导思想,旨在探索通过系统性地复用与传递已有知识资产以提升深度模型构建效率的方法论。基于对模型构建过程的分析,研究从知识获取、结构承载与参数实现三个关键环节出发,分别探讨了知识迁移思想在不同维度上的具体应用与实践。

	在知识层面,针对少样本场景下模型对标注数据高度依赖且知识蒸馏稳定性差的问题,本文提出了 Prompt-Distiller 方法。该方法通过融合双教师知识与对比学习策略,实现了低资源条件下从大型提示学习模型到小型学生模型的鲁棒知识迁移,显著提升了数据匮乏环境下的模型构建效率与性能 。

	在结构层面,为解决神经架构搜索 (Neural Architecture Search, NAS) 成本高昂且架构经验难以跨越异构搜索空间复用的瓶颈,本文构建了 \textsc{Bridge} 框架。该框架通过统一的神经架构表示学习与跨域映射机制,结合进化序贯迁移优化(ESTO)策略,首次实现了跨异构搜索空间的架构知识有效迁移,大幅降低了自动化模型设计的开销 。

	在参数层面,聚焦于参数融合中存在的稳定性与性能保持难题,本文提出了 KG-MFTO 范式。该方法利用知识图谱显式建模模型关系,并通过课程规划与知识引导的进化求解器,在零训练条件下实现了多个大型语言模型(LLM)参数的高效、稳定融合,为模型能力的即时、低成本集成提供了新的解决方案 。

	本文通过 Prompt-Distiller、\textsc{Bridge} 和 KG-MFTO 这三项分别聚焦于知识、结构、参数层面的独立研究工作,为深度模型的高效构建提供了具体的技术手段,验证了知识迁移思想在应对不同效率瓶颈方面的有效性与潜力,为推动深度学习向更高效、可持续的方向发展贡献了有益的探索。
\end{cabstract}
% 中文关键词,请使用英文逗号分隔:
\ckeywords{深度学习;高效人工智能;知识迁移;小样本学习;知识蒸馏;神经架构搜索;模型融合}

\begin{eabstract}	% 英文摘要
	Deep learning has achieved remarkable success across numerous domains, yet the enhancement of model performance is often accompanied by a dramatic increase in data, computational, and human costs. This has led to a critical "scale-efficiency" contradiction, hindering the technology's widespread application and sustainable development. To address this challenge, this dissertation adopts "Knowledge Transfer" as the core guiding principle, aiming to explore methodologies for improving the efficiency of deep model construction by systematically reusing and transferring existing knowledge assets. Based on an analysis of the model construction process, this research investigates the specific applications and practices of the knowledge transfer concept within three key dimensions: knowledge acquisition, structure bearing, and parameter implementation, exploring each independently.

	At the knowledge layer, addressing the high dependency on labeled data and the instability of knowledge distillation in few-shot scenarios, this dissertation proposes the Prompt-Distiller method. By integrating dual-teacher knowledge and a contrastive learning strategy, Prompt-Distiller achieves robust knowledge transfer from large prompt-based learning models to small student models under low-resource conditions, significantly enhancing model construction efficiency and performance in data-scarce environments .

	At the structure layer, to tackle the high computational cost of Neural Architecture Search (NAS) and the difficulty of reusing architectural experience across heterogeneous search spaces, the \textsc{Bridge} framework is constructed. Through unified neural architecture representation learning, cross-domain mapping mechanisms, and an Evolutionary Sequential Transfer Optimization (ESTO) strategy, \textsc{Bridge} enables effective architectural knowledge transfer across heterogeneous search spaces for the first time, substantially reducing the overhead of automated model design .

	At the parameter layer, focusing on the stability and performance retention challenges in parameter merging, the KG-MFTO paradigm is proposed. This method utilizes a knowledge graph to explicitly model inter-model relationships and employs curriculum planning along with a knowledge-guided evolutionary solver to achieve efficient and stable merging of parameters from multiple Large Language Models (LLMs) under zero-training conditions, offering a novel solution for instant, low-cost capability integration .

	Through these three independent research efforts—Prompt-Distiller, \textsc{Bridge}, and KG-MFTO—focused respectively on the knowledge, structure, and parameter layers, this dissertation provides concrete technical approaches for efficient deep model construction. It validates the effectiveness and potential of the knowledge transfer principle in addressing diverse efficiency bottlenecks, contributing valuable explorations toward advancing deep learning in a more efficient and sustainable direction.
\end{eabstract}
% 英文关键词,请使用英文逗号分隔,关键词内可以空格:
\ekeywords{Deep Learning; Efficient Artificial Intelligence; Knowledge Transfer; Few-Shot Learning; Knowledge Distillation; Neural Architecture Search; Model Merging}

% 封面和摘要配置完成
% 封面部分
\makecover

\frontmatter %%%前置部分(封面后绪论前)
% \cquauthpage[contents/cover1.pdf]
% \cquauthpage[contents/cover2.pdf]
%\cquauthpage[contents/cover3.pdf]
%\cquauthpage[contents/cover4.pdf]

%% 原创声明和授权说明书,可选:用扫描页替换
%\cquauthpage[authscan.pdf]
%\cquauthpage

% 摘要
\makeabstract

%% 目录,注意需要多次编译才能更新
\setlength{\cftbeforetoctitleskip}{0pt}
\setlength{\cftaftertoctitleskip}{20pt}
\tableofcontents


% \setlength{\cftbeforelottitleskip}{0pt}
% \setlength{\cftafterlottitleskip}{20pt}
%% 插图索引,可选,如不用可注释掉
% \renewcommand*{\listfigurename}{图目录}
\clearpage
\phantomsection
\addcontentsline{toc}{chapter}{插图索引}
\listoffigures
\listoffiguresEN
\setlength{\cftbeforelottitleskip}{0pt}
\setlength{\cftafterlottitleskip}{20pt}
%% 表格索引,可选
% \renewcommand*{\listtablename}{表目录}
\clearpage
\phantomsection
\addcontentsline{toc}{chapter}{表格索引}
\listoftables
\listoftablesEN
%% 公式索引,可选
% \listofequations
% \listofequationsEN
%% 符号对照表,可选
\clearpage
% \addcontentsline{toc}{chapter}{主要符号对照表}
\phantomsection
\chapterstar{主要符号对照表}

% \renewcommand{\arraystretch}{1.2}
\fancyhead[R]{\small\songti 主要符号对照表}
\thispagestyle{fancy}
\begin{xltabular}{\textwidth}{>{\raggedright\arraybackslash}p{0.2\textwidth} >{\raggedright\arraybackslash}p{0.28\textwidth} X}
	% \toprule
	\textbf{符号}                                                                           & \textbf{名称}   & \textbf{解释}                                                \\
	% \midrule
	\endfirsthead

	% \toprule
	\endhead

	% \bottomrule
	\endfoot

	\endlastfoot

	$\mathcal{T}$                                                                         & 任务            & 从分布 $p(\mathcal{T})$ 中采样的任务                       \\
	$p(\mathcal{T})$                                                                      & 任务分布          & 任务的先验分布                \\
	$\mathcal{S}$, $\mathcal{Q}$                                                          & 支持集/查询集       & 少样本任务的训练支持集与用于评估的查询集                                      \\
	$\mathcal{Y}^q$                                                                       & 查询集标签         & 查询集 $\mathcal{Q}$ 的真实标签集合                                 \\
	$f_\theta$                                                                            & 参数化模型         & 参数为 $\theta$ 的模型或学习器                                      \\
	$\theta$                                                                              & 模型参数          & 可学习的模型参数                                                  \\
	$\mathbb{E}[\cdot]$                                                                   & 期望算子          & 在分布上取数学期望                                                 \\
	$\mathcal{L}$                                                                         & 损失函数          & 训练/评估中使用的目标函数                                             \\
	$\Theta_T,\,\Theta_S$                                                                 & 教师/学生参数       & 教师与学生模型的参数                                                \\
	$\Theta_{T'}$                                                                         & 教师预训练参数       & 教师在任务微调前的预训练参数                                            \\
	$s_{\Theta}(x)$                                                                       & Logits        & 输入 $x$ 上的未归一化输出分数                                         \\
	$T$                                                                                   & 温度系数          & Softmax 温度,用于生成更平滑的软分布                                    \\
	$p_T,\,p_S$                                                                           & 教师/学生分布       & 教师/学生的软概率分布                      \\
	$D_{\mathrm{KL}}(\cdot\Vert\cdot)$                                                    & KL 散度         & Kullback--Leibler 散度,度量分布差异                               \\
	$\mathcal{L}_{\mathrm{KD}},\,\mathcal{L}_{\mathrm{MLM}},\,\mathcal{L}_{\mathrm{CPD}}$ & 蒸馏/MLM/对比蒸馏损失 & 知识蒸馏、掩码语言建模与对比蒸馏等损失项                                      \\
	$\lambda_j$                                                                           & 损失权重          & 各损失项的加权系数                                                 \\
	$X,\,\tilde{X}$                                                                       & 监督/无标注样本集     & 少样本训练集与用于蒸馏的无标注数据                                         \\
	$\mathcal{A}$                                                                         & 搜索空间          & NAS 候选架构集合                                                \\
	$a\in\mathcal{A}$                                                                     & 架构            & 待搜索的网络结构                                                  \\
	$w,\,\mathcal{W}_a$                                                                   & 权重/权重空间       & 架构权重与其可行集合                                                \\
	$\mathcal{L}_{\mathrm{train}},\,\mathcal{L}_{\mathrm{val}}$                           & 训练/验证损失       & NAS 内外层优化目标 \\
	$\bm{x}$                                                                              & 输入架构序列        & 神经架构的序列化表示                                           \\
	$\bm{z}$                                                                              & 潜变量           & VAE 的隐变量                                                  \\
	$q_{\phi}(\bm{z}\mid\bm{x})$                                                          & 编码器           & VAE 近似后验分布                                                \\
	$p_{\theta}(\bm{x}\mid\bm{z})$                                                        & 解码器           & VAE 生成分布                                                  \\
	$\mathcal{L}_{\phi,\theta}(\bm{x})$                                                   & 证据下界(ELBO)   & VAE 的优化目标,含重建项与 KL 正则                                     \\
	$\bm{m}$                                                                              & 记忆向量          & 序列记忆 / \texttt{CLS} 隐状态                                     \\
	$L$                                                                                   & 序列长度          & 生成/解码的最大序列长度                                              \\
	$\mathcal{M}=\{M_1,\dots,M_n\}$                                                       & 专家模型集合        & 待融合的专长 LLM 集合                                             \\
	$\tau_j\in\mathcal{T}$                                                                & 来自任务集合的下游任务            & 评估任务或基准                                                   \\
	$\mathrm{Merge}(\mathcal{M},\boldsymbol{\alpha})$                                     & 融合算子          & 按权重 $\boldsymbol{\alpha}$ 融合模型参数                          \\
	$\boldsymbol{\alpha}$                                                                 & 融合权重          & 模型融合的系数向量                                                 \\
	$\mathcal{F}$                                                                         & 评估函数          & 汇总融合模型在多任务上的表现                                            \\
	$w_j$                                                                                 & 任务权重          & 评估函数中各任务的权重                                               \\
	$\mathcal{C}$                                                                         & 约束空间          & 融合权重的可行域(如单纯形)                                            \\
	$\Delta^{n-1}$                                                                        & 单纯形             & 融合权重 $\boldsymbol{\alpha}$ 的标准约束域                              \\
	$f=(\mathcal{M}_f,\mathcal{T}_f)$                                                     & 形式(Form)     & 子问题的模型/任务子集                                               \\
	$\ell$                                                                                & 课程级别          & 形式的难度等级                                                   \\
	$G=(V,E)$                                                                             & 知识图谱          & 模型与任务关系的异构图                                               \\
	$s_{ii'}$                                                                             & 协同分数          & 模型 $M_i$ 与 $M_{i'}$ 的协同强度                                 \\
	$u_{ii'}$                                                                             & 不确定度          & 模型对关系的不确定度                                                \\
	$\pi_{ij}$                                                                            & 模型–任务性能       & 模型 $M_i$ 在任务 $\tau_j$ 上的得分                                \\
	$\hat{y}_f$                                                                           & 预测评分          & GNN 预测的形式性能                                               \\
	$\hat{\boldsymbol{\alpha}}_f$                                                         & 预测权重          & GNN 预测的融合系数                                               \\
	$\hat{S}_f$                                                                           & 预测协同矩阵        & GNN 预测的形式内协同关系                                            \\
	$\boldsymbol{\alpha}_f^{\star}$                                                       & 最优权重          & 形式 $f$ 的最优融合系数                                            \\
	$g_f(i,i')$                                                                           & 协同增益          & 形式内模型对的协同增益量                                              \\
	$R$                                                                                   & 相关矩阵          & 由协同矩阵导出的相关性矩阵                                             \\
	$\sigma(f)$                                                                           & 不确定性聚合        & 形式的不确定性度量                                                 \\
	$\rho$                                                                                & 秩相关系数         & Spearman 秩相关系数                                            \\
	% \bottomrule
	% (按出现顺序到此结束)
\end{xltabular}

% %% 缩略语对照表,可选
\clearpage
\phantomsection
\addcontentsline{toc}{chapter}{缩略语对照表}
% \chapter*{缩略语对照表}

\begin{abbreviate}[0pt][50pt]
	\item[BERT]          Bidirectional Encoder Representations from Transformers,  双向编码表示
	\item[CDAN]          Conditional Domain Adversarial Network,                   条件域对抗网络
	\item[CDP]           Cross-Domain Predictor,                                    跨域性能预测器
	\item[CMA-ES]        Covariance Matrix Adaptation Evolution Strategy,          协方差矩阵自适应进化策略
	\item[CLS]           Classification Token,                                      分类标记(CLS 标记)
	\item[CNN]           Convolutional Neural Network,                             卷积神经网络
	\item[CORAL]         Correlation Alignment,                                    相关性对齐
	\item[DA]            Domain Adaptation,                                        领域自适应
	\item[DAG]           Directed Acyclic Graph,                                   有向无环图
	\item[DANN]          Domain-Adversarial Neural Network,                        领域对抗神经网络
	\item[DARE-TIES]     DARE-TIES,                                                强化版 TIES 融合算法
	\item[DARTS]         Differentiable Architecture Search,                       可微架构搜索
	\item[DNN]           Deep Neural Network,                                      深度神经网络
	\item[ELBO]          Evidence Lower Bound,                                     证据下界(VAE 目标)
	\item[EMT-NAS]       Evolutionary Multi-Task NAS,                               演化多任务神经架构搜索
	\item[EOMMR]         Evolutionary Optimization for Model Merging,              基于演化的模型融合优化
	\item[ESTO]          Evolutionary Sequential Transfer Optimization,            演化序贯迁移优化
	\item[FF]            Feed-Forward Network,                                     前馈网络
	\item[FSL]           Few-Shot Learning,                                        少样本学习
	\item[GA]            Genetic Algorithm,                                        遗传算法
	\item[GAT]           Graph Attention Network,                                  图注意力网络
	\item[GNN]           Graph Neural Network,                                     图神经网络
	\item[GPT]           Generative Pre-trained Transformer,                       生成式预训练 Transformer
	\item[KD]            Knowledge Distillation,                                   知识蒸馏
	\item[KG]            Knowledge Graph,                                          知识图谱
	\item[KG-MFTO]       Knowledge-Guided Multi-Form Optimization,                 知识引导的多形式优化
	\item[KL]            Kullback--Leibler Divergence,                             KL 散度
	\item[LAION]         Large-scale Artificial Intelligence Open Network,         大规模开放人工智能网络
	\item[LLM]           Large Language Model,                                     大型语言模型
	\item[LN]            Layer Normalization,                                      层归一化
	\item[LoRA]          Low-Rank Adaptation,                                      低秩适配
	\item[MDS]           Multi-Dimensional Scaling,                                多维尺度分析
	\item[MergeKit]      MergeKit(Model Merging Toolkit),                        模型融合工具集
	\item[MFTO]          Multi-Form Optimization,                                  多形式优化
	\item[MHA]           Multi-Head Attention,                                     多头注意力
	\item[MLM]           Masked Language Modeling,                                 掩码语言建模
	\item[MMD]           Maximum Mean Discrepancy,                                 最大均值差异
	\item[MoE]           Mixture of Experts,                                       混合专家
	\item[MSA]           Multi-Head Self-Attention,                                多头自注意力
	\item[NAS]           Neural Architecture Search,                               神经架构搜索
	\item[OOE]           Operation On Edge,                                        边上算子
	\item[OON]           Operation On Node,                                        节点上算子
	\item[PEFT]          Parameter-Efficient Fine-Tuning,                          参数高效微调
	\item[PET]           Pattern-Exploiting Training,                              模式利用训练
	\item[PLM]           Pre-trained Language Model,                               预训练语言模型
	\item[PRR]           Performance Retention Rate,                               性能保留率
	\item[ReLU]          Rectified Linear Unit,                                    线性整流单元
	\item[RKD]           Relational Knowledge Distillation,                        关系蒸馏
	\item[RNN]           Recurrent Neural Network,                                 循环神经网络
	\item[SFT]           Supervised Fine-Tuning,                                   监督微调
	\item[SLERP]         Spherical Linear Interpolation,                           球面线性插值
	\item[TIES]          TIES-Merging,                                              冲突掩码与符号对齐的融合算法
	\item[TNAS]          Transferable Neural Architecture Search,                  可迁移神经架构搜索
	\item[UCB]           Upper Confidence Bound,                                    上置信界
	\item[VAE]           Variational Autoencoder,                                  变分自编码器
	\item[ViT]           Vision Transformer,                                       视觉 Transformer(ViT)

\end{abbreviate}



\mainmatter %%% 主体部分(绪论开始,结论为止)
%* 子文件的多少和内容由你决定(最好以章为单位),基本原则是提速预览、脉络清晰、管理容易。

% 设置字号为小四
\renewcommand{\normalsize}{\fontsize{12pt}{20pt}\selectfont}
% 设置小四正文行间距为 20 磅
\setstretch{1.312}

\subfile{contents/1-introduction_cleaned}
\subfile{contents/2-related-work_cleaned}
\subfile{contents/4-bridge-nas_cleaned}
\subfile{contents/3-prompt-distiller_cleaned}
\subfile{contents/5-kg-mfto-merge_cleaned}
\subfile{contents/6-conclusion_cleaned}

\backmatter %%% 后置部分(致谢、参考文献、附录等)

%% 参考文献
% 顺序编码制:cqunumerical
% 注意:至少需要引用一篇参考文献,否则下面两行会引起编译错误。
% \bibliographystyle{cqunumerical}
\bibliographystyle{gbt7714-numerical_new}
\bibliography{ref/thesis,ref/thesis-ext}


%% 附录(按ABC...分节,证明、推导、程序、个人简历等)
\appendix
\subfile{contents/7-appendix_cleaned}


%% 致谢
\subfile{contents/8-ack_cleaned}

\end{document}
