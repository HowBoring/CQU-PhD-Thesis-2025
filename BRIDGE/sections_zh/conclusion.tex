\documentclass[../main_zh.tex]{subfiles}

\begin{document}

\section{结论}\label{sec:conclusion}

\DiffBegin
本文介绍了 \OUR{},一个通过原则性的跨域知识迁移来推进演化 TNAS 的新颖框架。通过系统地利用来自异构域的架构模式,\OUR{} 能够在不同的搜索空间之间高效迁移高性能解决方案,同时通过学习到的潜在表示减少计算开销。我们的综合实证评估表明,\OUR{} 在 NAS-Bench-201、NAS-Bench-101 和 DARTS 搜索空间中,相较于最先进的 NAS 方法,在峰值和平均准确率上均取得了显著的提升,表现出持续的优势。通过流形可视化和性能分布度量的量化分析验证了该框架的有效性。

尽管 \OUR{} 为 NAS 中的跨域知识迁移奠定了有前景的基础,但仍有几个关键挑战有待进一步研究。主要的局限性在于,在扩展到复杂的异构搜索空间时,标记的架构数据本身就非常稀缺。未来的研究方向包括探索自监督学习方法,以解决少样本或零样本场景。此外,减轻不同领域之间的负迁移仍然是一个悬而未决的挑战。关键的研究领域包括:开发基于领域相似性度量的自适应迁移机制,建立稳健的源领域选择标准,以及设计用于不良迁移模式的早期检测系统。
此外,架构搜索空间的指数级增长影响了探索和知识迁移过程。一个有前景的研究方向是,将搜索空间的分层分解与在域内和跨域操作的多目标优化框架相结合。这些进步将扩展 \OUR{} 的能力,为资源受限的部署在各种现实世界应用中实现稳健且计算高效的演化 NAS。
\DiffEnd

\end{document}