\documentclass[../main.tex]{subfiles}

\begin{document}

\section{Conclusions}\label{sec:conclusion}

\DiffBegin
This paper introduces \OUR{}, a novel framework that advances evolutionary TNAS through principled cross-domain knowledge transfer. By systematically leveraging architectural patterns from heterogeneous domains, \OUR{} enables efficient migration of high-performing solutions across distinct search spaces while reducing computational overhead through learned latent representations. Our comprehensive empirical evaluation demonstrates \OUR{}'s consistent advantages over state-of-the-art NAS methods, achieving substantial improvements in both peak and mean accuracy across NAS-Bench-201, NAS-Bench-101, and DARTS search spaces. Quantitative analysis through manifold visualization and performance distribution metrics validates the framework's efficacy.

While \OUR{} establishes a promising foundation for cross-domain knowledge transfer in NAS, several key challenges warrant further investigation. The primary limitation stems from the inherent scarcity of labeled architecture data when scaling to complex, heterogeneous search spaces. Future research directions include exploring self-supervised learning approaches to address few- or zero-shot scenarios. Furthermore, mitigating negative transfer between dissimilar domains remains an open challenge. Critical areas for investigation include developing adaptive transfer mechanisms grounded in domain similarity metrics, establishing robust criteria for source domain selection, and designing early detection systems for adverse transfer patterns.
Additionally, the exponential growth of the architecture search space impacts both the exploration and knowledge transfer processes. A promising research direction involves hierarchical decomposition of the search space coupled with multi-objective optimization frameworks that operate both within and across domains. These advances will extend \OUR{}'s capabilities, enabling robust and computationally efficient evolutionary NAS for resource-constrained deployments across diverse real-world applications.
\DiffEnd

\end{document}